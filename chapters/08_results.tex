\clearpage
%\newcommand{\tablescale}{0.9}

%\section{RESULTADOS}
%    Los resultados se ven reflejados en los cuadros durante los 6 meses de internado en el Hospital para Animales de la Facultad Integral Defensores del Chaco.

%\subsection{Cuadros Clínicos}

%\subsubsection*{Cuadro Nº 1: Enfermedades virales}
%\vspace{-0.5cm}
%\pdft{0.696}{1}{statics/cuadros.pdf}{0cm 2.8cm 0cm 2.9cm}{true}

% 01_viral.tex
\begin{landscape}
	\section{RESULTADOS}
	Los resultados se ven reflejados en los cuadros durante los 6 meses de internado en el Hospital para Animales de la Facultad Integral Defensores del Chaco.

	% =============================================================================
	% CONFIGURACIÓN (basada en tu modelo)
	% =============================================================================
	\setlength{\tabcolsep}{1pt}
	\setlength{\extrarowheight}{3pt}
	\fontsize{10pt}{10pt}\selectfont
	%\footnotesize

	% =============================================================================
	% CÁLCULO DE ANCHOS DE COLUMNAS
	% =============================================================================
	% Total de ancho disponible: \linewidth (en landscape es el ancho horizontal)
	% Restamos el espacio ocupado por líneas verticales y espacios entre columnas
	% 8 columnas = 7 líneas verticales + espacios entre columnas
	\newlength{\availablewidth}
	\setlength{\availablewidth}{\linewidth}
	\addtolength{\availablewidth}{-14\tabcolsep} % 7 espacios internos * 2
	\addtolength{\availablewidth}{-0.4pt} % Ancho de líneas verticales

	% =============================================================================
	% LONGTABLE CON CAPTION AL PRINCIPIO
	% =============================================================================
	\begin{longtable}{
		|>{\centering\arraybackslash}m{0.08\availablewidth}           % ESPECIE
		|>{\centering\arraybackslash}m{0.07\availablewidth}           % Nº CASOS
		|>{\centering\arraybackslash}m{0.12\availablewidth}           % ENFERMEDADES
		|>{\raggedright\arraybackslash}m{0.12\availablewidth}          % FÁRMACO
		|>{\raggedright\arraybackslash}m{0.24\availablewidth}          % PRINCIPIO ACTIVO
		|>{\raggedright\arraybackslash}m{0.152\availablewidth}          % DOSIS/VÍA
		|>{\centering\arraybackslash}m{0.06\availablewidth}            % TIEMPO
		|>{\raggedright\arraybackslash}m{0.15\availablewidth}|         % OBSERVACIONES
	}

		% ──────────────── ENCABEZADO Y CAPTION DE LA PRIMERA PÁGINA ────────────────
		\caption{\raggedright Registro de Tratamientos para Enfermedades Virales}
		\label{tab:virales} \\
		\hline
		\rowcolor{blue!10}
		\textbf{ESPECIE} & \textbf{Nº DE CASOS} & \textbf{ENFERMEDAD} & \textbf{FÁRMACO} & 
		\textbf{PRINCIPIO ACTIVO} & \textbf{DOSIS/VÍA DE ADMINISTRACIÓN} & 
		\textbf{DURA\-CIÓN} & \textbf{OBSERVACIONES} \\
		\hline
		\endfirsthead
		
		% ──────────────── ENCABEZADO PARA PÁGINAS SIGUIENTES (SOLUCIÓN) ────────────────
		% Se usa \multicolumn para el título de continuación
		\multicolumn{8}{l}{\normalsize\textbf{(Continuación) Cuadro \ref{tab:virales}}. Registro de Tratamientos para Enfermedades Virales} \\[6pt]
		\hline
		\rowcolor{blue!10}
		\textbf{ESPECIE} & \textbf{Nº DE CASOS} & \textbf{ENFERMEDAD} & \textbf{FÁRMACO} & 
		\textbf{PRINCIPIO ACTIVO} & \textbf{DOSIS/VÍA DE ADMINISTRACIÓN} & 
		\textbf{DURA\-CIÓN} & \textbf{OBSERVACIONES} \\
		\hline
		\endhead

		% ──────────────── PIE DE PÁGINA (para páginas intermedias) ────────────────
		\hline
		\multicolumn{8}{r}{\footnotesize\textit{Continúa en la siguiente página \ldots}}
		\endfoot

		% ──────────────── PIE FINAL (para la última página) ────────────────
		\hline
		\multicolumn{8}{l}{\footnotesize\textit{\textbf{Fuente}: Elaboración propia.}}
		\endlastfoot

		% =============================================================================
		% CONTENIDO DE LA TABLA (extraído del PDF '01_viral.pdf')
		% =============================================================================

		% Canino – Parvo virus
		\multirow{7}{*}{\textbf{Canino}} & \multirow{7}{*}{14} & \multirow{7}{*}{Parvo virus} &
		Ringer Lactato & Electrolitos & Según criterio medico IV & \multirow{7}{*}{3-5 días} & \multirow{7}{=}{\parbox{\linewidth}{2 pacientes no respondieron al tratamiento}} \\
		\cline{4-6}
		& & & Ditral & Sulfadoxina, trimetoprima & 1ml/10kg IV-IM & & \\
		\cline{4-6}
		& & & Clortetrasone & Oxitetraciclina clorhidrato, florfenicol, prednisolona, lidocaina & 1ml/5kg IM & & \\
		\cline{4-6}
		& & & Desalgina & Dipirona & 1ml/10kg IV-IM-SC & & \\
		\cline{4-6}
		& & & Complejo B & Vitamina B1, Vitamina B2, Vitamina B6, Vitamina B12, Vitamina B15 & 1ml/20kg IM-SC & & \\
		\cline{4-6}
		& & & AminoLab Forte & Aminoacidos, dextrosa, cloruro de potasio, cloruro de sodio, cloruro de calcio, acetato de sodio, vitamina B1 B2 B6 B12 & 1ml/1kg IV & & \\
		\cline{4-6}
		& & & Ranitidina & Ranitidina & 1ml/5kg IV-IM & & \\
		\hline

		% Canino – Distemper
		\multirow{6}{*}{\textbf{Canino}} & \multirow{6}{*}{5} & \multirow{6}{*}{Distemper} &
		Ringer Lactato & Electrolitos & Según criterio médico IV & \multirow{6}{*}{3-5 días} & \multirow{6}{=}{\parbox{\linewidth}{1 paciente no respondió al tratamiento}} \\
		\cline{4-6}
		& & & Desalgina & Dipirona & 1ml/10kg IV-IM-SC & & \\
		\cline{4-6}
		& & & AminoLab Forte & Aminoacidos, dextrosa, cloruro de potasio, cloruro de sodio, cloruro de calcio, acetato de sodio, vitamina B1 B2 B6 B12 & 1ml/1kg IV & & \\
		\cline{4-6}
		& & & Hepato-Ject & Acido tioctico, acido orotico, DL-Metionina/N-Acetil-L-Metionina, D-pantenol, Piridoxina HCl, acido folico, cloruro de colina, inotisol, betaina HCl & 1ml/20kg IV-IM & & \\
		\cline{4-6}
		& & & Floxa Bencetimida & Enrofloxacina y clorhidrato de bencetimida & 1ml/10kg IM & & \\
		\hline

		% Canino – Hepatitis
		\multirow{4}{*}{\textbf{Canino}} & \multirow{4}{*}{7} & \multirow{4}{*}{Hepatitis} &
		Complejo B & Vitamina B1, Vitamina B2, Vitamina B6, Vitamina B12, Vitamina B15 & 1ml/20kg IM-SC & \multirow{4}{*}{3-5 días} & \multirow{4}{*}{Satisfactorio} \\
		\cline{4-6}
		& & & Clortetrasone & Oxitetraciclina clorhidrato, florfenicol, prednisolona, lidocaina & 1ml/5kg IM & & \\
		\cline{4-6}
		& & & Mercepton & Acetil D-L-Metionina, cloruro de colina, Inositol, Vitamina B1 B2 B6 B12, nicotinmida, pantotenato de calcio & 1ml/10kg IV-IM-SC & & \\
		\hline

		% Felino – Panleucopenia
		\multirow{4}{*}{\textbf{Felino}} & \multirow{4}{*}{1} & \multirow{4}{*}{Panleucopenia} &
		Ringer Lactato & Electrolitos & Según criterio medico IV & \multirow{4}{*}{3-5 días} & \multirow{4}{*}{Satisfactorio} \\
		\cline{4-6}
		& & & Complejo B & Vitamina B1, Vitamina B2, Vitamina B6, Vitamina B12, Vitamina B15 & 1ml/20kg IM-SC & & \\
		\cline{4-6}
		& & & Clortetrasone & Oxitetraciclina clorhidrato, florfenicol, prednisolona, lidocaina & 1ml/5kg IM & & \\
		\cline{4-6}
		& & & Ditral & Sulfadoxina, trimetoprima & 1ml/10kg IV-IM & & \\
		\hline
	\end{longtable}
\end{landscape}


\clearpage


%\subsubsection*{Cuadro Nº 2: Enfermedades bacterianas}
%\vspace{0.5cm}
%\pdft{0.699}{2}{statics/cuadros.pdf}{0cm 3cm 0cm 5.5cm}{true}

% 02_bacteriana.tex
\begin{landscape}

	%\section*{REGISTRO DE ENFERMEDADES BACTERIANAS}
	%\vspace{0.3cm}

	% =============================================================================
	% CONFIGURACIÓN (basada en tu modelo)
	% =============================================================================
	\setlength{\tabcolsep}{1pt}
	\setlength{\extrarowheight}{3pt}
	\fontsize{10pt}{10pt}\selectfont

	% =============================================================================
	% CÁLCULO DE ANCHOS DE COLUMNAS
	% =============================================================================
	% Total de ancho disponible: \linewidth (en landscape es el ancho horizontal)
	% Restamos el espacio ocupado por líneas verticales y espacios entre columnas
	% 8 columnas = 7 líneas verticales + espacios entre columnas
	%\newlength{\availablewidth}
	\setlength{\availablewidth}{\linewidth}
	\addtolength{\availablewidth}{-14\tabcolsep} % 7 espacios internos * 2
	\addtolength{\availablewidth}{-0.4pt} % Ancho de líneas verticales

	% =============================================================================
	% LONGTABLE CON CAPTION AL PRINCIPIO
	% =============================================================================
	\begin{longtable}{
		|>{\centering\arraybackslash}m{0.07\availablewidth}			% ESPECIE
		|>{\centering\arraybackslash}m{0.07\availablewidth}           % Nº CASOS
		|>{\centering\arraybackslash}m{0.14\availablewidth}           % ENFERMEDADES
		|>{\raggedright\arraybackslash}m{0.12\availablewidth}          % FÁRMACO
		|>{\raggedright\arraybackslash}m{0.24\availablewidth}          % PRINCIPIO ACTIVO
		|>{\raggedright\arraybackslash}m{0.152\availablewidth}          % DOSIS/VÍA
		|>{\centering\arraybackslash}m{0.06\availablewidth}            % TIEMPO
		|>{\raggedright\arraybackslash}m{0.14\availablewidth}|         % OBSERVACIONES
	}

		% ──────────────── Encabezado y CAPTION COMO TÍTULO (solo en primera página) ────────────────
		\caption{\raggedright Registro de Tratamientos para Enfermedades Bacterianas}
		\label{tab:bacterianas} \\
		\hline
		\rowcolor{blue!10}
		\textbf{ESPECIE} & \textbf{Nº DE CASOS} & \textbf{ENFERMEDADES} & \textbf{FÁRMACO} & 
		\textbf{PRINCIPIO ACTIVO} & \textbf{DOSIS/VÍA DE ADMINISTRACIÓN} & 
		\textbf{DURA\-CIÓN} & \textbf{OBSERVACIONES} \\
		\hline
		\endfirsthead

		% ──────────────── ENCABEZADO PARA PÁGINAS SIGUIENTES ────────────────		
		\multicolumn{8}{l}{\normalsize\textbf{(Continuación) Cuadro \ref{tab:bacterianas}}. Registro de Tratamientos para Enfermedades Virales} \\[6pt]
		\hline
		\rowcolor{blue!10}
		\textbf{ESPECIE} & \textbf{Nº DE CASOS} & \textbf{ENFERMEDADES} & \textbf{FÁRMACO} & 
		\textbf{PRINCIPIO ACTIVO} & \textbf{DOSIS/VÍA DE ADMINISTRACIÓN} & 
		\textbf{DURA\-CIÓN} & \textbf{OBSERVACIONES} \\
		\hline
		\endhead

		% ──────────────── PIE DE PÁGINA (continuación) ────────────────
		\hline
		\multicolumn{8}{r}{\footnotesize\textit{Continúa en la siguiente página \ldots}}
		\endfoot

		% ──────────────── PIE FINAL ────────────────
		\hline
		\multicolumn{8}{l}{\footnotesize\textit{\textbf{Fuente}: Elaboración propia.}}
		\endlastfoot

		% =============================================================================
		% CONTENIDO DE LA TABLA (extraído del PDF '02_bacteriana.pdf')
		% =============================================================================

		% Canino – Traqueo-bronquitis
		\multirow{4}{*}{\textbf{Canino}} & \multirow{4}{*}{15} & \multirow{4}{*}{Traqueo-bronquitis} &
		Complejo yodado & Yoduro de potasio, yodo metálico & 0,3ml/10kg IM & \multirow{4}{*}{3-5 días} & \multirow{4}{*}{Satisfactorio} \\
		\cline{4-6}
		& & & Bromexan & Brohexina & 1-3ml IM & & \\
		\cline{4-6}
		& & & Desalgina & Dipirona & 1ml/10kg IV-IM-SC & & \\
		\cline{4-6}
		& & & Interflox & Erofloxacino & 1ml/10kg IM-SC & & \\
		\hline

		% Canino – Gastroenteritis
		\multirow{4}{*}{\textbf{Canino}} & \multirow{4}{*}{29} & \multirow{4}{*}{Gastroenteritis} &
		Gentamicin & Gentamicina, metilparabeno, propilparabeno & 1ml/20kg IV-IM & \multirow{4}{*}{3-5 días} & \multirow{4}{*}{Satisfactorio} \\
		\cline{4-6}
		& & & AminoLab forte & Aminoacidos, dextrosa, cloruro de potasio, cloruro de sodio, cloruro de calcio, acetato de sodio, vitamina B1 B2 B6 B12 & 1ml/1kg IV & & \\
		\cline{4-6}
		& & & Complejo B & Vitamina B1, Vitamina B2, Vitamina B6, Vitamina B12, Vitamina B15 & 1ml/20kg IM-SC & & \\
		\cline{4-6}
		& & & Hepato-Ject & Acido tioctico, acido orotico, DL-Metionina/N-Acetil-L-Metionina, D-pantenol, Piridoxina HCl, acido folico, cloruro de colina, inotisol, betaina HCl & 1ml/20kg IV-IM & & \\
		\hline

		% Canino – Conjuntivitis
		\textbf{Canino} & 10 & Conjuntivitis & Dexaflox & Colirio, ofloxacina, dexametazona fosfato & 1-2 gotas V.Oftálmica & 3-5 días & Satisfactorio \\
		\hline

		% Canino – Otitis
		\multirow{2}{*}{\textbf{Canino}} & \multirow{2}{*}{7} & \multirow{2}{*}{Otitis} &
		Atriben & Triamcinolona & 1ml/20kg IM-SC & \multirow{2}{*}{3-5 días} & \multirow{2}{*}{Satisfactorio} \\
		\cline{4-6}
		& & & Opter & Enrofloxacina, clotrimazol, betametazona, ivermectina & 6-12 gotas V.Ótica & & \\
		\hline

		% Felino – Traqueo-bronquitis
		\multirow{3}{*}{\textbf{Felino}} & \multirow{3}{*}{3} & \multirow{3}{*}{Traqueo-bronquitis} &
		Complejo yodado & Yoduro de potasio, yodo metálico & 0,3ml/10kg IM & \multirow{3}{*}{3-5 días} & \multirow{3}{*}{Satisfactorio} \\
		\cline{4-6}
		& & & Bromexan & Brohexina & 1-3ml IM & & \\
		\cline{4-6}
		& & & Interflox & Erofloxacino & 1ml/10kg IM-SC & & \\
		\hline

		% Felino – Conjuntivitis
		\textbf{Felino} & 4 & Conjuntivitis & Dexaflox & Colirio, ofloxacina, dexametazona fosfato & 1-2 gotas V.Oftálmica & 3-5 días & Satisfactorio \\
		\hline

		% Felino – Otitis
		\multirow{2}{*}{\textbf{Felino}} & \multirow{2}{*}{1} & \multirow{2}{*}{Otitis} &
		Atriben & Triamcinolona & 1ml/20kg IM-SC & \multirow{2}{*}{3-5 días} & \multirow{2}{*}{Satisfactorio} \\
		\cline{4-6}
		& & & Opter & Enrofloxacina, clotrimazol, betametazona, ivermectina & 6-12 gotas V.Ótica & & \\
		\hline

	\end{longtable}
\end{landscape}

\clearpage

% 03_metabolica.tex
\begin{landscape}

%\section*{REGISTRO DE ENFERMEDADES METABÓLICAS}
%\vspace{0.3cm}

% =============================================================================
% CONFIGURACIÓN (basada en tu modelo)
% =============================================================================
\setlength{\tabcolsep}{1pt}
\fontsize{10pt}{10pt}\selectfont
\setlength{\extrarowheight}{3pt}

% =============================================================================
% LONGTABLE CON CAPTION AL PRINCIPIO
% =============================================================================
\begin{longtable}{
  |>{\centering\arraybackslash}m{0.07\linewidth}           % ESPECIE
  |>{\centering\arraybackslash}m{0.07\linewidth}           % Nº CASOS
  |>{\centering\arraybackslash}m{0.14\linewidth}           % ENFERMEDADES
  |>{\raggedright\arraybackslash}m{0.12\linewidth}          % FÁRMACO
  |>{\raggedright\arraybackslash}m{0.26\linewidth}          % PRINCIPIO ACTIVO
  |>{\raggedright\arraybackslash}m{0.14\linewidth}          % DOSIS/VÍA
  |>{\centering\arraybackslash}m{0.06\linewidth}            % TIEMPO
  |>{\raggedright\arraybackslash}m{0.12\linewidth}|         % OBSERVACIONES
}

% ──────────────── CAPTION COMO TÍTULO (solo en primera página) ────────────────
\caption{Registro de Tratamientos para Enfermedades Metabólicas}
\label{tab:metabolicas}
\endfirsthead

\caption[]{(Continuación) Registro de Tratamientos para Enfermedades Metabólicas}

% ──────────────── ENCABEZADO PARA PÁGINAS SIGUIENTES ────────────────
\hline
\rowcolor{blue!10}
\textbf{ESPECIE} & \textbf{Nº DE CASOS} & \textbf{ENFERMEDADES} & \textbf{FÁRMACO} & 
\textbf{PRINCIPIO ACTIVO} & \textbf{DOSIS/VÍA DE ADMINISTRACIÓN} & 
\textbf{DURA\-CIÓN} & \textbf{OBSERVACIO\-NES} \\
\hline
\endhead

% ──────────────── PIE DE PÁGINA (continuación) ────────────────
\hline
\multicolumn{8}{r}{\footnotesize\textit{Continúa en la siguiente página \ldots}}
\endfoot

% ──────────────── PIE FINAL ────────────────
\hline
\multicolumn{8}{l}{\footnotesize\textit{\textbf{Fuente}: Elaboración propia.}}
\endlastfoot

% ──────────────── ENCABEZADO PRIMERA PÁGINA (con color) ────────────────
\hline
\rowcolor{blue!10}
\textbf{ESPECIE} & \textbf{Nº DE CASOS} & \textbf{ENFERMEDADES} & \textbf{FÁRMACO} & 
\textbf{PRINCIPIO ACTIVO} & \textbf{DOSIS/VÍA DE ADMINISTRACIÓN} & 
\textbf{DURA\-CIÓN} & \textbf{OBSERVACIO\-NES} \\
\hline

% =============================================================================
% CONTENIDO DE LA TABLA (extraído del PDF '03_metabolica.pdf')
% =============================================================================

% Canino – Ascitis
\multirow{3}{*}{\textbf{Canino}} & \multirow{3}{*}{3} & \multirow{3}{*}{Ascitis} &
Furozur & Furosemida, dexametasona, fosfato de sodio & 1ml/10kg IM & \multirow{3}{*}{3-5 días} & \multirow{3}{*}{Satisfactorio} \\
\cline{4-6}
& & & Cefotaxima & Cefotaxima sódica & 2,5ml/10kg IV-IM & & \\
\cline{4-6}
& & & AminoLab forte & Aminoácidos, dextrosa, cloruro de potasio, cloruro de sodio, cloruro de calcio, acetato de sodio, vitamina B1 B2 B6 B12 & 1ml/1kg IV & & \\
\hline

% Canino – Anemia
\multirow{4}{*}{\textbf{Canino}} & \multirow{4}{*}{2} & \multirow{4}{*}{Anemia} &
Ringer Lactato & Electrolitos & Según criterio médico IV & \multirow{4}{*}{3-5 días} & \multirow{4}{*}{Satisfactorio} \\
\cline{4-6}
& & & AminoLab forte & Aminoácidos, dextrosa, cloruro de potasio, cloruro de sodio, cloruro de calcio, acetato de sodio, vitamina B1 B2 B6 B12 & 1ml/1kg IV & & \\
\cline{4-6}
& & & Complejo B & Vitamina B1, Vitamina B2, Vitamina B6, Vitamina B12, Vitamina B15 & 1ml/20kg IM-SC & & \\
\cline{4-6}
& & & Duplafer & Complejo hierro dextano, vitamina B12 & 1-3ml IM & & \\
\hline

% Felino – Anemia
\multirow{4}{*}{\textbf{Felino}} & \multirow{4}{*}{1} & \multirow{4}{*}{Anemia} &
Ringer Lactato & Electrolitos & Según criterio médico IV & \multirow{4}{*}{3-5 días} & \multirow{4}{*}{Satisfactorio} \\
\cline{4-6}
& & & AminoLab forte & Aminoácidos, dextrosa, cloruro de potasio, cloruro de sodio, cloruro de calcio, acetato de sodio, vitamina B1 B2 B6 B12 & 1ml/1kg IV & & \\
\cline{4-6}
& & & Complejo B & Vitamina B1, Vitamina B2, Vitamina B6, Vitamina B12, Vitamina B15 & 1ml/20kg IM-SC & & \\
\cline{4-6}
& & & Duplafer & Complejo hierro dextano, vitamina B12 & 1-3ml IM & & \\
\hline

\end{longtable}
\end{landscape}

\clearpage

%\subsubsection*{Cuadro Nº 4: Enfermedades parasitarias}
%\vspace{-0.5cm}
%\pdft{0.699}{4}{statics/cuadros.pdf}{0cm 9cm 0cm 9.7cm}{true}

% 04_parasitaria.tex
\begin{landscape}

	%\section*{REGISTRO DE ENFERMEDADES PARASITARIAS}
	%\vspace{0.3cm}

	% =============================================================================
	% CONFIGURACIÓN (basada en tu modelo)
	% =============================================================================
	\setlength{\tabcolsep}{1pt}
	\setlength{\extrarowheight}{3pt}
	\fontsize{10pt}{10pt}\selectfont

	% =============================================================================
	% CÁLCULO DE ANCHOS DE COLUMNAS
	% =============================================================================
	% Total de ancho disponible: \linewidth (en landscape es el ancho horizontal)
	% Restamos el espacio ocupado por líneas verticales y espacios entre columnas
	% 8 columnas = 7 líneas verticales + espacios entre columnas
	%\newlength{\availablewidth}
	\setlength{\availablewidth}{\linewidth}
	\addtolength{\availablewidth}{-14\tabcolsep} % 7 espacios internos * 2
	\addtolength{\availablewidth}{-0.4pt} % Ancho de líneas verticales

	% =============================================================================
	% LONGTABLE CON CAPTION AL PRINCIPIO
	% =============================================================================
	\begin{longtable}{
		|>{\centering\arraybackslash}m{0.08\availablewidth}           % ESPECIE
		|>{\centering\arraybackslash}m{0.08\availablewidth}           % Nº CASOS
		|>{\centering\arraybackslash}m{0.14\availablewidth}           % ENFERMEDADES
		|>{\raggedright\arraybackslash}m{0.10\availablewidth}          % FÁRMACO
		|>{\raggedright\arraybackslash}m{0.24\availablewidth}          % PRINCIPIO ACTIVO
		|>{\raggedright\arraybackslash}m{0.152\availablewidth}          % DOSIS/VÍA
		|>{\centering\arraybackslash}m{0.06\availablewidth}            % TIEMPO
		|>{\raggedright\arraybackslash}m{0.14\availablewidth}|         % OBSERVACIONES
	}

		% ──────────────── CAPTION COMO TÍTULO (solo en primera página) ────────────────
		\caption{\raggedright Registro de Tratamientos para Enfermedades Parasitarias}
		\label{tab:parasitarias} \\
		\hline
		\rowcolor{blue!10}
		\textbf{ESPECIE} & \textbf{Nº DE CASOS} & \textbf{ENFERMEDADES} & \textbf{FÁRMACO} & 
		\textbf{PRINCIPIO ACTIVO} & \textbf{DOSIS/VÍA DE ADMINISTRACIÓN} & 
		\textbf{DURA\-CIÓN} & \textbf{OBSERVACIONES} \\
		\hline
		\endfirsthead

		% ──────────────── ENCABEZADO PARA PÁGINAS SIGUIENTES ────────────────
		\multicolumn{8}{l}{\normalsize\textbf{(Continuación) Cuadro \ref{tab:parasitarias}}. Registro de Tratamientos para Enfermedades Parasitarias} \\[6pt]
		\hline
		\rowcolor{blue!10}
		\textbf{ESPECIE} & \textbf{Nº DE CASOS} & \textbf{ENFERMEDADES} & \textbf{FÁRMACO} & 
		\textbf{PRINCIPIO ACTIVO} & \textbf{DOSIS/VÍA DE ADMINISTRACIÓN} & 
		\textbf{DURA\-CIÓN} & \textbf{OBSERVACIONES} \\
		\hline
		\endhead

		% ──────────────── PIE DE PÁGINA (continuación) ────────────────
		\hline
		\multicolumn{8}{r}{\footnotesize\textit{Continúa en la siguiente página \ldots}}
		\endfoot

		% ──────────────── PIE FINAL ────────────────
		\hline
		\multicolumn{8}{l}{\footnotesize\textit{\textbf{Fuente}: Elaboración propia.}}
		\endlastfoot

		% =============================================================================
		% CONTENIDO DE LA TABLA (extraído del PDF '04_parasitaria.pdf')
		% =============================================================================

		% Canino – Sarna sarcoptica
		\multirow{6}{*}{\textbf{Canino}} & \multirow{6}{*}{17} & \multirow{6}{*}{Sarna sarcoptica} &
		Ranger & Ivermectina 1\% & 1ml/30kg SC & \multirow{6}{*}{3 días} & \multirow{6}{*}{Satisfactorio} \\
		\cline{4-6}
		& & & Galerfin & Clorfeniramina maleto & 0,2-2ml IV-IM & & \\
		\cline{4-6}
		& & & Atriben & Triamcinolona & 1ml/20kg IM-SC & & \\
		\cline{4-6}
		& & & Cefavet & Cefalexina & 1ml/20kg IM-SC & & \\
		\cline{4-6}
		& & & Dermo vet & Benzoato de bencilo, boro salicílico & Tópico & & \\
		\cline{4-6}
		& & & Impacto & Cipermetrina, clorpirifos, citronelal & 1ml/1litro de agua & & \\
		\hline

		% Canino – Sarna demodecica
		\multirow{5}{*}{\textbf{Canino}} & \multirow{5}{*}{1} & \multirow{5}{*}{Sarna demodecica} &
		Ranger & Ivermectina 1\% & 1ml/30kg SC & \multirow{5}{*}{3 días} & \multirow{5}{*}{Satisfactorio} \\
		\cline{4-6}
		& & & Atriben & Triamcinolona & 1ml/20kg IM-SC & & \\
		\cline{4-6}
		& & & Cefavet & Cefalexina & 1ml/20kg IM-SC & & \\
		\cline{4-6}
		& & & Dermo vet & Benzoato de bencilo, boro salicílico & Tópico & & \\
		\cline{4-6}
		& & & Impacto & Cipermetrina, clorpirifos, citronelal & 1ml/1litro de agua & & \\
		\hline

	\end{longtable}
\end{landscape}

\clearpage

%\subsubsection*{Cuadro Nº 5: Intoxicaciones}
%\vspace{-0.5cm}
%\pdft{0.699}{5}{statics/cuadros.pdf}{0cm 10cm 0cm 10cm}{true}

% 05_intoxicaniones.tex
\begin{landscape}

%\section*{REGISTRO DE INTOXICACIONES}
%\vspace{0.3cm}

% =============================================================================
% CONFIGURACIÓN (basada en tu modelo)
% =============================================================================
\setlength{\tabcolsep}{1pt}
\setlength{\extrarowheight}{3pt}
\fontsize{10pt}{10pt}\selectfont

% =============================================================================
	% CÁLCULO DE ANCHOS DE COLUMNAS
	% =============================================================================
	% Total de ancho disponible: \linewidth (en landscape es el ancho horizontal)
	% Restamos el espacio ocupado por líneas verticales y espacios entre columnas
	% 8 columnas = 7 líneas verticales + espacios entre columnas
	%\newlength{\availablewidth}
	\setlength{\availablewidth}{\linewidth}
	\addtolength{\availablewidth}{-14\tabcolsep} % 7 espacios internos * 2
	\addtolength{\availablewidth}{-0.4pt} % Ancho de líneas verticales

% =============================================================================
% LONGTABLE CON CAPTION AL PRINCIPIO
% =============================================================================
\begin{longtable}{
	|>{\centering\arraybackslash}m{0.07\availablewidth}           % ESPECIE
	|>{\centering\arraybackslash}m{0.07\availablewidth}           % Nº CASOS
	|>{\centering\arraybackslash}m{0.14\availablewidth}           % ENFERMEDADES
	|>{\raggedright\arraybackslash}m{0.12\availablewidth}          % FÁRMACO
	|>{\raggedright\arraybackslash}m{0.24\availablewidth}          % PRINCIPIO ACTIVO
	|>{\raggedright\arraybackslash}m{0.152\availablewidth}          % DOSIS/VÍA
	|>{\centering\arraybackslash}m{0.06\availablewidth}            % TIEMPO
	|>{\raggedright\arraybackslash}m{0.14\availablewidth}|         % OBSERVACIONES
}

% ──────────────── CAPTION COMO TÍTULO (solo en primera página) ────────────────
\caption{\raggedright Registro de Tratamientos para Intoxicación Alimenticia}
\label{tab:intoxicaciones} \\
\hline
\rowcolor{blue!10}
\textbf{ESPECIE} & \textbf{Nº DE CASOS} & \textbf{ENFERMEDADES} & \textbf{FÁRMACO} & 
\textbf{PRINCIPIO ACTIVO} & \textbf{DOSIS/VÍA DE ADMINISTRACIÓN} & 
\textbf{DURA\-CIÓN} & \textbf{OBSERVACIONES} \\
\hline
\endfirsthead

% ──────────────── ENCABEZADO PARA PÁGINAS SIGUIENTES ────────────────
\multicolumn{8}{l}{\normalsize\textbf{(Continuación) Cuadro \ref{tab:intoxicaciones}}. Registro de Tratamientos para Intoxicación Alimenticia}\\[6pt]
\hline
\rowcolor{blue!10}
\textbf{ESPECIE} & \textbf{Nº DE CASOS} & \textbf{ENFERMEDADES} & \textbf{FÁRMACO} & 
\textbf{PRINCIPIO ACTIVO} & \textbf{DOSIS/VÍA DE ADMINISTRACIÓN} & 
\textbf{DURA\-CIÓN} & \textbf{OBSERVACIONES} \\
\hline
\endhead

% ──────────────── PIE DE PÁGINA (continuación) ────────────────
\hline
\multicolumn{8}{r}{\footnotesize\textit{Continúa en la siguiente página \ldots}}
\endfoot

% ──────────────── PIE FINAL ────────────────
\hline
\multicolumn{8}{l}{\footnotesize\textit{\textbf{Fuente}: Elaboración propia.}}
\endlastfoot

% =============================================================================
% CONTENIDO DE LA TABLA (extraído del PDF '05_intoxicaniones.pdf')
% =============================================================================

% Canino – Intoxicación alimenticia
\multirow{5}{*}{\textbf{Canino}} & \multirow{5}{*}{13} & \multirow{5}{*}{\parbox{\linewidth}{Intoxicación alimenticia}} &
Suero fisiológico & Cloruro de sodio & Según criterio médico IV & \multirow{5}{*}{3 días} & \multirow{5}{*}{Satisfactorio} \\
\cline{4-6}
& & & Mercepton & Acetil D-L-Metionina, cloruro de colina, Inositol, Vitamina B1 B2 B6 B12, nicotinmida, pantotenato de calcio & 1ml/10kg IV-IM-SC & & \\
\cline{4-6}
& & & Desalgina & Dipirona & 1ml/10kg IV-IM-SC & & \\
\cline{4-6}
& & & Hepato-Ject & Acido tioctico, acido orotico, DL-Metionina/N-Acetil-L-Metionina, D-pantenol, Piridoxina HCl, acido folico, cloruro de colina, inotisol, betaina HCl & 1ml/20kg IV-IM & & \\
\cline{4-6}
& & & Atropina & Sulfato de atropina & 0,5ml/10kg IV-IM-SC & & \\
\hline

\end{longtable}
\end{landscape}

\clearpage

%\subsubsection*{Cuadro Nº 6: Traumatismos}
%\vspace{-0.5cm}
%\pdft{0.699}{6}{statics/cuadros.pdf}{0cm 8cm 0cm 8cm}{true}

% 06_tramatismo.tex
\begin{landscape}

	%\section*{REGISTRO DE TRAUMATISMOS}
	%\vspace{0.3cm}

	% =============================================================================
	% CONFIGURACIÓN (basada en tu modelo)
	% =============================================================================
	\setlength{\tabcolsep}{1pt}
	\setlength{\extrarowheight}{3pt}
	\fontsize{10pt}{10pt}\selectfont

	% =============================================================================
	% CÁLCULO DE ANCHOS DE COLUMNAS
	% =============================================================================
	% Total de ancho disponible: \linewidth (en landscape es el ancho horizontal)
	% Restamos el espacio ocupado por líneas verticales y espacios entre columnas
	% 8 columnas = 7 líneas verticales + espacios entre columnas
	\newlength{\availablewidth}
	\setlength{\availablewidth}{\linewidth}
	\addtolength{\availablewidth}{-14\tabcolsep} % 7 espacios internos * 2
	\addtolength{\availablewidth}{-0.4pt} % Ancho de líneas verticales

	% =============================================================================
	% LONGTABLE CON CAPTION AL PRINCIPIO
	% =============================================================================
	\begin{longtable}{
		|>{\centering\arraybackslash}m{0.08\availablewidth}           % ESPECIE
		|>{\centering\arraybackslash}m{0.07\availablewidth}           % Nº CASOS
		|>{\centering\arraybackslash}m{0.14\availablewidth}           % ENFERMEDADES
		|>{\raggedright\arraybackslash}m{0.12\availablewidth}          % FÁRMACO
		|>{\raggedright\arraybackslash}m{0.24\availablewidth}          % PRINCIPIO ACTIVO
		|>{\raggedright\arraybackslash}m{0.152\availablewidth}          % DOSIS/VÍA
		|>{\centering\arraybackslash}m{0.07\availablewidth}            % TIEMPO
		|>{\raggedright\arraybackslash}m{0.12\availablewidth}|         % OBSERVACIONES
	}

		% ──────────────── CAPTION COMO TÍTULO (solo en primera página) ────────────────
		\caption{\raggedright Registro de Tratamientos para Traumatismos}
		\label{tab:traumatismos} \\
		\hline
		\rowcolor{blue!10}
		\textbf{ESPECIE} & \textbf{Nº DE CASOS} & \textbf{ENFERMEDADES} & \textbf{FÁRMACO} & 
		\textbf{PRINCIPIO ACTIVO} & \textbf{DOSIS/VÍA DE ADMINISTRACIÓN} & 
		\textbf{DURA\-CIÓN} & \textbf{OBSERVACIO\-NES} \\
		\hline
		\endfirsthead

		% ──────────────── ENCABEZADO PARA PÁGINAS SIGUIENTES ────────────────
		\multicolumn{8}{l}{\normalsize\textbf{(Continuación) Cuadro \ref{tab:traumatismos}}. Registro de Tratamientos para Traumatismos} \\[6pt]
		\hline
		\rowcolor{blue!10}
		\textbf{ESPECIE} & \textbf{Nº DE CASOS} & \textbf{ENFERMEDADES} & \textbf{FÁRMACO} & 
		\textbf{PRINCIPIO ACTIVO} & \textbf{DOSIS/VÍA DE ADMINISTRACIÓN} & 
		\textbf{DURA\-CIÓN} & \textbf{OBSERVACIO\-NES} \\
		\hline
		\endhead

		% ──────────────── PIE DE PÁGINA (continuación) ────────────────
		\hline
		\multicolumn{8}{r}{\footnotesize\textit{Continúa en la siguiente página \ldots}}
		\endfoot

		% ──────────────── PIE FINAL ────────────────
		\hline
		\multicolumn{8}{l}{\footnotesize\textit{\textbf{Fuente}: Elaboración propia.}}
		\endlastfoot

		% =============================================================================
		% CONTENIDO DE LA TABLA (extraído del PDF '06_tramatismo.pdf')
		% =============================================================================

		% Canino – Traumatismo (fractura)
		\multirow{7}{*}{\textbf{Canino}} & \multirow{7}{*}{4} & \multirow{7}{*}{\parbox{\linewidth}{\centering{Traumatismo\\(fractura)}}} &
		Xilacina & Xilacina & 1ml/20kg IV-IM & \multirow{7}{*}{\parbox{\linewidth}{3-4 semanas}} &  \\
		\cline{4-6}
		& & & Lidocaina & Lidocaina & Según criterio médico & & \\
		\cline{4-6}
		& & & Oximed & Oxitetraciclina, bencidamina & 1ml/5kg IM-SC & & \\
		\cline{4-6}
		& & & Desalgina & Dipirona & 1ml/10kg IV-IM-SC & & \\
		\cline{4-6}
		& & & Agua oxigenada & Peróxido de hidrógeno & Tópico & & \\
		\cline{4-6}
		& & & Yodo podovidona & Complejo de yodo molecular con podovidona & Tópico & & \\
		\cline{4-6}
		& & & Galmetrim plus & Cipermetrina, imidacloprid, sulfadiazina de plata & Local externo & & \\
		\hline

		% Canino – Politraumatismo
		\multirow{4}{*}{\textbf{Canino}} & \multirow{4}{*}{24} & \multirow{4}{*}{Politraumatismo} &
		Desalgina & Dipirona & 1ml/10kg IV-IM-SC & \multirow{4}{*}{3-5 días} & \multirow{4}{*}{Satisfactorio} \\
		\cline{4-6}
		& & & Oximed & Oxitetraciclina, bencidamina & 1ml/5kg IM-SC & & \\
		\cline{4-6}
		& & & Complejo B & Vitamina B1, Vitamina B2, Vitamina B6, Vitamina B12, Vitamina B15 & 1ml/20kg IM-SC & & \\
		\cline{4-6}
		& & & Diclofenaco & Diclofenaco & 1ml/20kg IV-IM & & \\
		\hline

		% Felino – Contusión
		\multirow{4}{*}{\textbf{Felino}} & \multirow{4}{*}{2} & \multirow{4}{*}{Contusión} &
		Ankofen & Ketoprofeno & 0,5ml/25kg IM & \multirow{4}{*}{3-5 días} & \multirow{4}{*}{Satisfactorio} \\
		\cline{4-6}
		& & & Agua oxigenada & Peróxido de hidrógeno & Tópico & & \\
		\cline{4-6}
		& & & Yodo podovidona & Complejo de yodo molecular con podovidona & Tópico & & \\
		\cline{4-6}
		& & & Emplasto cicatrizante & Ácido galotanico, podovidona, yodo metálico, yodo de potasio & Tópico & & \\
		\hline

	\end{longtable}
\end{landscape}


\clearpage

%\subsubsection*{Cuadro Nº 7: Cirugías}
%\vspace{-0.5cm}
%\pdft{0.699}{7}{statics/cuadros.pdf}{0cm 10cm 0cm 0cm}{true}
\begin{landscape}

	%\subsection*{Cuadro Nº 7: Cirugías}
	%\vspace{0.3cm}
	% =============================================================================
	% CONFIGURACIÓN TIPOGRÁFICA (NO TOCAR – FUNCIONA PERFECTAMENTE)
	% =============================================================================
	%\fontsize{10pt}{15pt}\selectfont
	%\setlength{\extrarowheight}{2pt}
	%\setlength{\tabcolsep}{3pt}
	\setlength{\tabcolsep}{1pt}
	\setlength{\extrarowheight}{3pt}
	\fontsize{10pt}{10pt}\selectfont

	% =============================================================================
	% CÁLCULO DE ANCHOS DE COLUMNAS
	% =============================================================================
	% Total de ancho disponible: \linewidth (en landscape es el ancho horizontal)
	% Restamos el espacio ocupado por líneas verticales y espacios entre columnas
	% 8 columnas = 7 líneas verticales + espacios entre columnas
	%\newlength{\availablewidth}
	\setlength{\availablewidth}{\linewidth}
	\addtolength{\availablewidth}{-14\tabcolsep} % 7 espacios internos * 2
	\addtolength{\availablewidth}{-0.4pt} % Ancho de líneas verticales

	% =============================================================================
	% TABLA
	% =============================================================================
	\begin{longtable}{
		|>{\centering\arraybackslash}m{0.08\availablewidth}    %Especie
		|>{\centering\arraybackslash}m{0.07\availablewidth}    %Casos
		|>{\centering\arraybackslash}m{0.10\availablewidth}    %Cirugia
		|>{\raggedright\arraybackslash}m{0.16\availablewidth}  %Farmaco
		|>{\raggedright\arraybackslash}m{0.22\availablewidth}  %Principio
		|>{\raggedright\arraybackslash}m{0.12\availablewidth}  %Dosis
		|>{\raggedright\arraybackslash}m{0.12\availablewidth}  %Via
		|>{\raggedright\arraybackslash}m{0.122\availablewidth}| %Observaciones
	}

		% -----------------------------------------------------------------------------
		% CAPTION Y LABEL
		% -----------------------------------------------------------------------------
		\caption{\raggedright Registro de Cirugías realizadas durante el periodo de internado}
		\label{tab:cirugias} \\
		\hline
		\rowcolor{blue!10}
		\textbf{ESPECIE} & \textbf{Nº DE CASOS} & \textbf{CIRUGÍA} &
		\textbf{FÁRMACO} & \textbf{PRINCIPIO ACTIVO} & \textbf{DOSIS} &
		\textbf{VÍA DE ADMINISTRACIÓN} & \textbf{OBSERVACIO\-NES} \\
		\hline
		\endfirsthead

		% -----------------------------------------------------------------------------
		% ENCABEZADO (PRIMERA PÁGINA Y SIGUIENTES)
		% -----------------------------------------------------------------------------
		\multicolumn{8}{l}{\normalsize\textbf{(Continuación) Cuadro \ref{tab:cirugias}}. Registro de Cirugías realizadas durante el periodo de internado} \\[6pt]
		\hline
		\rowcolor{blue!10}
		\textbf{ESPECIE} & \textbf{Nº DE CASOS} & \textbf{CIRUGÍA} &
		\textbf{FÁRMACO} & \textbf{PRINCIPIO ACTIVO} & \textbf{DOSIS} &
		\textbf{VÍA DE ADMINISTRACIÓN} & \textbf{OBSERVACIO\-NES} \\
		\hline
		\endhead

		% -----------------------------------------------------------------------------
		% PIE DE CONTINUACIÓN
		% -----------------------------------------------------------------------------
		\hline
		\multicolumn{8}{r}{\footnotesize\textit{Continúa en la siguiente página \ldots}}
		\endfoot

		\hline
		\multicolumn{8}{l}{\footnotesize\textit{\textbf{Fuente}: Elaboración propia.}}
		\endlastfoot

		% =============================================================================
		% CANINO – ORQUIECTOMÍA
		% =============================================================================
		\multirow{9}{*}{\textbf{Canino}} &
		\multirow{9}{*}{3} &
		\multirow{9}{*}{Orquiectomía} &
		Xilacina & Xilacina & 1\,ml/20\,kg & IV-IM & \\
		\cline{4-7}
		& & & Lidocaína & Lidocaína & Según criterio médico & IM-SC & \\
		\cline{4-7}
		& & & Keta-A & Ketamina 10\% & 1\,ml/10\,kg & IV-IM & \\
		\cline{4-7}
		& & & Cefavet & Cefalexina & 1\,ml/20\,kg & IM-SC & \\
		\cline{4-7}
		& & & Galmetrim plus & Cipermetrina, imidacloprid, sulfadiazina de plata & Según criterio médico & Local externo & \\
		\cline{4-7}
		& & & Agua oxigenada & Peróxido de hidrógeno & Según criterio médico & Tópico & \\
		\cline{4-7}
		& & & Yodo podovidona & Complejo de yodo molecular con podovidona & Según criterio médico & Tópico & \\
		\cline{4-7}
		& & & Ácido poliglicólico 3/0 & Ácido poliglicólico & --- & --- & \\
		\cline{4-7}
		& & & Hilo externo & Lino retorcido & --- & --- & \\
		\hline

		% =============================================================================
		% CANINO – OVH
		% =============================================================================
		\multirow{17}{*}{\textbf{Canino}} &
		\multirow{17}{*}{7} &
		\multirow{17}{*}{OVH} &
		Xilacina & Xilacina & 1\,ml/20\,kg & IV-IM & \\
		\cline{4-7}
		& & & Lidocaína & Lidocaína & Según criterio médico & IM-SC & \\
		\cline{4-7}
		& & & Keta-A & Ketamina 10\% & 1\,ml/10\,kg & IV-IM & \\
		\cline{4-7}
		& & & Cefavet & Cefalexina & 1\,ml/20\,kg & IM-SC & \\
		\cline{4-7}
		& & & Galmetrim plus & Cipermetrina, imidacloprid, sulfadiazina de plata & Según criterio médico & Local externo & \\
		\cline{4-7}
		& & & Agua oxigenada & Peróxido de hidrógeno & Según criterio médico & Tópico & \\
		\cline{4-7}
		& & & Yodo podovidona & Complejo de yodo molecular con podovidona & Según criterio médico & Tópico & \\
		\cline{4-7}
		& & & Desalgina & Dipirona & 1\,ml/10\,kg & IV-IM-SC & \\
		\cline{4-7}
		& & & Atropina & Sulfato de atropina & 0,5\,ml/10\,kg & IV-IM-SC & \\
		\cline{4-7}
		& & & Antihem & Vitamina K & 2-5\,ml & IV & \\
		\cline{4-7}
		& & & Galerfin & Clorfeniramina maleato & 0,2-2\,ml & IV-IM & \\
		\cline{4-7}
		& & & Mercepton & Acetil D-L-Metionina, cloruro de colina, Inositol, Vitamina B1 B2 B6 B12, nicotinamida, pantotenato de calcio & 1\,ml/10\,kg & IV-IM-SC & \\
		\cline{4-7}
		& & & Ringer Lactato & Electrolitos & Según criterio médico & IV & \\
		\cline{4-7}
		& & & Ácido poliglicólico 2/0 & Ácido poliglicólico & --- & --- & \\
		\cline{4-7}
		& & & Ácido poliglicólico 3/0 & Ácido poliglicólico & --- & --- & \\
		\cline{4-7}
		& & & Hilo externo & Lino retorcido & --- & --- & \\
		\hline

		% =============================================================================
		% FELINO – ORQUIECTOMÍA
		% =============================================================================
		\multirow{9}{*}{\textbf{Felino}} &
		\multirow{9}{*}{9} &
		\multirow{9}{*}{Orquiectomía} &
		Xilacina & Xilacina & 1\,ml/20\,kg & IV-IM & \\
		\cline{4-7}
		& & & Lidocaína & Lidocaína & Según criterio médico & IM-SC & \\
		\cline{4-7}
		& & & Keta-A & Ketamina 10\% & 1\,ml/10\,kg & IV-IM & \\
		\cline{4-7}
		& & & Cefavet & Cefalexina & 1\,ml/20\,kg & IM-SC & \\
		\cline{4-7}
		& & & Galmetrim plus & Cipermetrina, imidacloprid, sulfadiazina de plata & Según criterio médico & Local externo & \\
		\cline{4-7}
		& & & Agua oxigenada & Peróxido de hidrógeno & Según criterio médico & Tópico & \\
		\cline{4-7}
		& & & Yodo podovidona & Complejo de yodo molecular con podovidona & Según criterio médico & Tópico & \\
		\cline{4-7}
		& & & Ácido poliglicólico 3/0 & Ácido poliglicólico & --- & --- & \\
		\cline{4-7}
		& & & Hilo externo & Lino retorcido & --- & --- & \\
		\hline

		% =============================================================================
		% FELINO – OVH
		% =============================================================================
		\multirow{17}{*}{\textbf{Felino}} &
		\multirow{17}{*}{14} &
		\multirow{17}{*}{OVH} &
		Xilacina & Xilacina & 1\,ml/20\,kg & IV-IM & \\
		\cline{4-7}
		& & & Lidocaína & Lidocaína & Según criterio médico & IM-SC & \\
		\cline{4-7}
		& & & Keta-A & Ketamina 10\% & 1\,ml/10\,kg & IV-IM & \\
		\cline{4-7}
		& & & Cefavet & Cefalexina & 1\,ml/20\,kg & IM-SC & \\
		\cline{4-7}
		& & & Galmetrim plus & Cipermetrina, imidacloprid, sulfadiazina de plata & Según criterio médico & Local externo & \\
		\cline{4-7}
		& & & Agua oxigenada & Peróxido de hidrógeno & Según criterio médico & Tópico & \\
		\cline{4-7}
		& & & Yodo podovidona & Complejo de yodo molecular con podovidona & Según criterio médico & Tópico & \\
		\cline{4-7}
		& & & Desalgina & Dipirona & 1\,ml/10\,kg & IV-IM-SC & \\
		\cline{4-7}
		& & & Atropina & Sulfato de atropina & 0,5\,ml/10\,kg & IV-IM-SC & \\
		\cline{4-7}
		& & & Antihem & Vitamina K & 2-5\,ml & IV & \\
		\cline{4-7}
		& & & Galerfin & Clorfeniramina maleato & 0,2-2\,ml & IV-IM & \\
		\cline{4-7}
		& & & Mercepton & Acetil D-L-Metionina, cloruro de colina, Inositol, Vitamina B1 B2 B6 B12, nicotinamida, pantotenato de calcio & 1\,ml/10\,kg & IV-IM-SC & \\
		\cline{4-7}
		& & & Ringer Lactato & Electrolitos & Según criterio médico & IV & \\
		\cline{4-7}
		& & & Ácido poliglicólico 2/0 & Ácido poliglicólico & --- & --- & \\
		\cline{4-7}
		& & & Ácido poliglicólico 3/0 & Ácido poliglicólico & --- & --- & \\
		\cline{4-7}
		& & & Hilo externo & Lino retorcido & --- & --- & \\
		\hline

	\end{longtable}
\end{landscape}

%\clearpage

%\subsubsection*{Cuadro Nº 8: Cirugías (Continuación)}
%\vspace{-0.5cm}
%\pdft{0.699}{8}{statics/cuadros.pdf}{0cm 5cm 0cm 5cm}{true}

\clearpage

%\subsubsection*{Cuadro Nº 9: Desparasitaciones}
%\vspace{-0.5cm}
%\pdft{0.699}{9}{statics/cuadros.pdf}{0cm 9cm 0cm 9cm}{true}

%\subsubsection*{Cuadro Nº 10: Inmunizaciones}
%\vspace{-0.5cm}
%\pdft{0.699}{10}{statics/cuadros.pdf}{0cm 11cm 0cm 11cm}{true}

%\subsubsection*{Cuadro Nº 11: Imagenología y otros servicios}
%\vspace{-0.5cm}
%\pdft{0.699}{11}{statics/cuadros.pdf}{0cm 11cm 0cm 11.7cm}{true}


%\subsection*{Cuadro Nº 8: Desparacitaciones}
%\vspace{0.3cm}

%\caption{Desparasitaciones y aplicación de vitaminas/minerales realizadas}%
%\label{tab:desparasitaciones}\\
%\endfirsthead

% 08_desparasitaciones.tex
\begin{landscape}

% =============================================================================
% CONFIGURACIÓN (basada en tu modelo)
% =============================================================================
\setlength{\tabcolsep}{1pt}
\fontsize{10pt}{10pt}\selectfont
\setlength{\extrarowheight}{3pt}

% =============================================================================
% LONGTABLE CON CAPTION AL PRINCIPIO
% =============================================================================
\begin{longtable}{
  |>{\centering\arraybackslash}m{0.08\linewidth}          % ESPECIE
  |>{\centering\arraybackslash}m{0.08\linewidth}           % Nº CASOS
  |>{\centering\arraybackslash}m{0.16\linewidth}           % PROCEDIMIENTO
  |>{\raggedright\arraybackslash}m{0.12\linewidth}          % FÁRMACO
  |>{\raggedright\arraybackslash}m{0.20\linewidth}          % PRINCIPIO ACTIVO
  |>{\raggedright\arraybackslash}m{0.14\linewidth}          % DOSIS/VÍA
  |>{\centering\arraybackslash}m{0.06\linewidth}           % DURACIÓN
  |>{\raggedright\arraybackslash}m{0.14\linewidth}|         % OBSERVACIONES
}

% ──────────────── CAPTION COMO TÍTULO (solo en primera página) ────────────────
\caption{Registro de Desparasitaciones y aplicación de vitaminas realizadas durante el internado}
\label{tab:desparasitaciones}
\endfirsthead

\caption[]{(Continuación) Registro de Desparasitaciones y aplicación de vitaminas realizadas durante el internado}

% ──────────────── ENCABEZADO PARA PÁGINAS SIGUIENTES ────────────────
\hline
\rowcolor{blue!10}
\textbf{ESPECIE} & \textbf{Nº DE CASOS} & \textbf{PROCEDIMIENTO} & \textbf{FÁRMACO} & 
\textbf{PRINCIPIO ACTIVO} & \textbf{DOSIS/VÍA DE ADMINISTRACIÓN} & 
\textbf{DURA\-CIÓN} & \textbf{OBSERVACIONES} \\
\hline
\endhead

% ──────────────── PIE DE PÁGINA (continuación) ────────────────
\hline
\multicolumn{8}{r}{\footnotesize\textit{continúa en la siguiente página \ldots}}
\endfoot

% ──────────────── PIE FINAL ────────────────
\hline
\multicolumn{8}{l}{\footnotesize\textit{\textbf{Fuente}: Elaboración propia.}}
\endlastfoot

% ──────────────── ENCABEZADO PRIMERA PÁGINA (con color) ────────────────
\hline
\rowcolor{blue!10}
\textbf{ESPECIE} & \textbf{Nº DE CASOS} & \textbf{PROCEDIMIENTO} & \textbf{FÁRMACO} & 
\textbf{PRINCIPIO ACTIVO} & \textbf{DOSIS/VÍA DE ADMINISTRACIÓN} & 
\textbf{DURA\-CIÓN} & \textbf{OBSERVACIONES} \\
\hline

% =============================================================================
% CONTENIDO DE LA TABLA (extraído del PDF '08_desparasitaciones.pdf')
% =============================================================================

% CANINO – Desparasitación externa
\multirow{2}{*}{\textbf{Canino}} & \multirow{2}{*}{8} & \multirow{2}{*}{\parbox{\linewidth}{Desparasitación externa}} &
Impacto & Cipermetrina, clorpirifos, citronelal & 1ml/1litro de agua & \multirow{2}{*}{1 mes} & \\
\cline{4-6}
& & & Galerfin & Clorfeniramina maleto & 0,2-2ml IV-IM & & \\
\hline

% CANINO – Desparasitación interna
\multirow{3}{*}{\textbf{Canino}} & \multirow{3}{*}{67} & \multirow{3}{*}{\parbox{\linewidth}{Desparasitación interna}} &
Levocan & Levamisol & 5 gotas/1kg Oral & \multirow{3}{*}{3 meses} & \\
\cline{4-6}
& & & Basken & Pamoato de pirantel, pamoato de oxantel & 1ml/1kg Oral & & \\
\cline{4-6}
& & & Vermic total & Praziquantel, pirantel pamoato, febantel & 1 tableta/10kg oral & & \\
\hline

% CANINO – Desparasitación interna-externa
\multirow{2}{*}{\textbf{Canino}} & \multirow{2}{*}{58} & \multirow{2}{*}{\parbox{\linewidth}{Desparasitación interna y externa}} &
Ranger & Ivermectina 1\% & 1ml/30kg SC & \multirow{2}{*}{3 meses} & \\
\cline{4-6}
& & & Galerfin & Clorfeniramina maleto & 0,2-2ml IV-IM & & \\
\hline

% CANINO – Aplicación de vitaminas y minerales
\multirow{2}{*}{\textbf{Canino}} & \multirow{2}{*}{125} & \multirow{2}{*}{\parbox{\linewidth}{Aplicación de vitaminas y minerales}} &
AminoLab Forte & Aminoácidos, dextrosa, cloruro de potasio, cloruro de sodio, cloruro de calcio, acetato de sodio, vitamina B1 B2 B6 B12 & 1ml/1kg IV & \multirow{2}{*}{\parbox{\linewidth}{Cada 15 días}} & \\
\cline{4-6}
& & & Complejo B & Vitamina B1, Vitamina B2, Vitamina B6, Vitamina B12, Vitamina B15 & 1ml/20kg IM-SC & & \\
\hline

\end{longtable}
\end{landscape}


\clearpage

%\subsection*{Cuadro Nº 9: Inmunizaciones}
%\vspace{0.3cm}

% 09_vacunaciones.tex
\begin{landscape}

% =============================================================================
% CONFIGURACIÓN (basada en tu modelo)
% =============================================================================
\setlength{\tabcolsep}{1pt}
\fontsize{10pt}{10pt}\selectfont
\setlength{\extrarowheight}{3pt}

% =============================================================================
% LONGTABLE CON CAPTION AL PRINCIPIO
% =============================================================================
\begin{longtable}{
  |>{\centering\arraybackslash}m{0.08\linewidth}           % ESPECIE
  |>{\centering\arraybackslash}m{0.08\linewidth}           % Nº DE CASOS
  |>{\raggedright\arraybackslash}m{0.26\linewidth}          % INMUNIZACION
  |>{\raggedright\arraybackslash}m{0.14\linewidth}          % FÁRMACO
  |>{\raggedright\arraybackslash}m{0.18\linewidth}          % PRINCIPIO ACTIVO
  |>{\centering\arraybackslash}m{0.14\linewidth}           % DOSIS/VÍA
  |>{\raggedright\arraybackslash}m{0.10\linewidth}|         % OBSERVACIONES
}

% ──────────────── CAPTION COMO TÍTULO (solo en primera página) ────────────────
\caption{Registro de Vacunaciones Aplicadas durante el periodo de internado}
\label{tab:vacunaciones}
\endfirsthead

\caption[]{(Continuación) Registro de Vacunaciones Aplicadas durante el periodo de internado}

% ──────────────── ENCABEZADO PARA PÁGINAS SIGUIENTES ────────────────
\hline
\rowcolor{blue!10}
\textbf{ESPECIE} & \textbf{Nº DE CASOS} & \textbf{INMUNIZACIÓN} & \textbf{FÁRMACO} & 
\textbf{PRINCIPIO ACTIVO} & \textbf{DOSIS/VÍA DE ADMINISTRACIÓN} & \textbf{OBSERVA\-CIONES} \\
\hline
\endhead

% ──────────────── PIE DE PÁGINA (continuación) ────────────────
\hline
\multicolumn{7}{r}{\footnotesize\textit{Continúa en la siguiente página \ldots}}
\endfoot

% ──────────────── PIE FINAL ────────────────
\hline
\multicolumn{7}{l}{\footnotesize\textit{\textbf{Fuente}: Elaboración propia.}}
\endlastfoot

% ──────────────── ENCABEZADO PRIMERA PÁGINA (con color) ────────────────
\hline
\rowcolor{blue!10}
\textbf{ESPECIE} & \textbf{Nº DE CASOS} & \textbf{INMUNIZACIÓN} & \textbf{FÁRMACO} & 
\textbf{PRINCIPIO ACTIVO} & \textbf{DOSIS/VÍA DE ADMINISTRACIÓN} & \textbf{OBSERVA\-CIONES} \\
\hline

% =============================================================================
% CONTENIDO DE LA TABLA (extraído del PDF '09_vacunaciones.pdf')
% =============================================================================

\multirow{3}{*}{\textbf{Canino}} & 97 & Parvovirus & Hipradog pv & Virus vivo modificado e inactivado & 1ml/animal SC & \\
\cline{2-7}
& 36 & Parvovirus, moquillo, hepatitis, laringotraqueitis, traqueobronquitis, parainfluenza, leptospirosis & Hipradog p7 Heptavalente & Virus vivo modificado e inactivado & 1ml/animal SC & \\
\cline{2-7}
& 1  & Rabia & Rabvac 3TK & Virus inactivado & 1ml/animal SC & \\
\hline
\textbf{Felino} & 5  & Panleucopenia, rinotraqueitis, calcivirus & Triple felina & Virus inactivado & 1ml/animal SC & \\
\hline

\end{longtable}
\end{landscape}

\clearpage

\begin{landscape}

\subsection*{Cuadro Nº 10: Imagenología y otros servicios}
\vspace{0.3cm}

\fontsize{10pt}{15pt}\selectfont
\setlength{\extrarowheight}{2pt}
\setlength{\tabcolsep}{3pt}

\begin{longtable}{
    |>{\centering\arraybackslash}m{0.10\linewidth}
    |>{\centering\arraybackslash}m{0.07\linewidth}
    |>{\raggedright\arraybackslash}m{0.40\linewidth}
    |>{\raggedright\arraybackslash}m{0.41\linewidth}|
}

% ================================
% CABECERA DE LA TABLA
% ================================
\hline
\rowcolor{blue!10}
    \textbf{ESPECIE}
    & \textbf{Nº DE CASOS}
    & \textbf{CASO/ACTIVIDAD}
    & \textbf{OBSERVACIONES} \\
\hline
\endhead

% -----------------------------------------------------------------------------
% CAPTION Y LABEL
% -----------------------------------------------------------------------------
\caption[]{Imagenología y otros servicios}
\label{tab:servicios}
\endlastfoot

% ================================
% DATOS - CANINO (filas fusionadas)
% ================================
\multirow{3}{*}{\centering\arraybackslash Canino} 
    & 26 
    & Rayos X
    & \\
\cline{2-4}
    
    & 7
    & Ecografía
    & \\
\cline{2-4}
    
    & 1 
    & Internación
    & \\
\hline

% ================================
% DATOS - FELINO
% ================================
\centering Felino 
    & 3
    & Internación
    & \\
\hline

\end{longtable}
\end{landscape}

\clearpage

% \begin{sidewaystable}
% \subsection{Análisis Global}
% \subsubsection{Cuadro Nº 12: Análisis Global de las Actividades}

% Esta consolidación refleja la aplicación práctica de los conocimientos adquiridos durante el internado, abarcando desde la medicina preventiva (Inmunizaciones, Desparasitaciones) y las habilidades técnicas (Cirugías) hasta el manejo de patologías específicas (Tratamientos).

% \paragraph{Resumen Global de Casuística por Tipo de Intervención}
% %\centering
% \setlength{\tabcolsep}{5pt}
% \renewcommand{\arraystretch}{1.5}
% \normalsize
% \definecolor{azulclinica}{RGB}{230, 240, 255}

% \begin{longtable}{|>{\raggedright\arraybackslash}p{0.22\textheight}
%                   |>{\raggedright\arraybackslash}p{0.33\textheight}
%                   |>{\centering\arraybackslash}p{0.18\textheight}
%                   |>{\centering\arraybackslash}p{0.18\textheight}|}
% \hline
% \rowcolor{azulclinica} % Azul claro para encabezado
% \textbf{Categoría Global} & \textbf{Detalle de Actividades Agrupadas} & \textbf{Total Casos} & \textbf{Porcentaje (\%)} \\
% \hline
% \endfirsthead

% \hline
% \rowcolor{blue!10}
% \textbf{Categoría Global} & \textbf{Detalle de Actividades Agrupadas} & \textbf{Total Casos} & \textbf{Porcentaje (\%)} \\
% \hline
% \endhead

% \hline
% \multicolumn{4}{|r|}{\footnotesize Continúa en la siguiente página\dots} \\
% \endfoot

% \hline
% \endlastfoot

% \textbf{Tratamientos} & Enfermedades Bacterianas, Virales, Parasitarias, Metabólicas, Traumatismos, Intoxicaciones, Vitaminas/Minerales. & 288 & 45.71\% \\
% \hline

% \textbf{Inmunizaciones (Vacunaciones)} & Vacunación Parvovirus, Heptavalente, Triple felina, Rabia. & 139 & 22.06\% \\
% \hline

% \textbf{Desparasitaciones} & Desparasitación interna, externa, interna y externa. & 133 & 21.11\% \\
% \hline

% \textbf{Imagenología y Servicios Auxiliares} & Rayos X, Ecografía e Internación de pacientes. & 37 & 5.87\% \\
% \hline

% \textbf{Cirugías} & OVH (Canina y Felina) y Orquiectomia (Canina y Felina). & 33 & 5.24\% \\
% \hline

% \hline
% \rowcolor{green!10} % Verde claro para el total
% \textbf{TOTAL GENERAL} & Suma de todas las actividades reportadas en el internado & \textbf{630} & \textbf{100.00\%} \\
% \hline
% \end{longtable}

% \caption{Resumen de actividades veterinarias realizadas durante el internado}
% \label{tab:actividades_veterinarias}
% \end{sidewaystable}

\begin{landscape}
%\subsection{Análisis Global}
%subsubsection{Cuadro Nº 12: Análisis Global de las Actividades}

%Esta consolidación refleja la aplicación práctica de los conocimientos adquiridos durante el internado, abarcando desde la medicina preventiva (Inmunizaciones, Desparasitaciones) y las habilidades técnicas (Cirugías) hasta el manejo de patologías específicas (Tratamientos).

%\paragraph{Resumen Global de Casuística por Tipo de Intervención}
%\centering
\setlength{\tabcolsep}{3pt}
%\fontsize{10pt}{12pt}\selectfont
\renewcommand{\arraystretch}{1.5}
\normalsize
%\definecolor{azulclinica}{RGB}{230, 240, 255}

%\setlength{\tabcolsep}{5pt}
%\fontsize{12pt}{15pt}\selectfont
%\setlength{\extrarowheight}{3pt}

\begin{longtable}{
    |>{\raggedright\arraybackslash}m{0.22\linewidth}	%Categoria
    |>{\raggedright\arraybackslash}m{0.4\linewidth}	%Detalle
    |>{\centering\arraybackslash}m{0.16\linewidth}		%Casos
    |>{\centering\arraybackslash}m{0.18\linewidth}|		%Porcentaje
}

	% ──────────────── CAPTION COMO TÍTULO (solo en primera página) ────────────────
	\caption{Resumen de actividades veterinarias realizadas durante el internado}
	\label{tab:actividades_veterinarias}
	\endfirsthead

	\caption[]{(Continuación) Resumen de actividades veterinarias realizadas durante el internado}

    % ──────────────── ENCABEZADO PARA PÁGINAS SIGUIENTES ────────────────
    \hline
    \rowcolor{blue!10}
    \textbf{Categoría Global} & \textbf{Detalle de Actividades Agrupadas} & \textbf{Total Casos} & \textbf{Porcentaje (\%)} \\
    \hline
    \endhead

    % ──────────────── PIE DE PÁGINA (continuación) ────────────────
    \hline
    \multicolumn{4}{r}{\footnotesize\textit{Continúa en la siguiente página\ldots}}
    \endfoot

    % ──────────────── PIE FINAL ────────────────
    \hline
    \multicolumn{4}{l}{\footnotesize\textit{\textbf{Fuente}: Elaboración propia.}}
    \endlastfoot

    % ──────────────── ENCABEZADO PRIMERA PÁGINA (con color) ────────────────
    \hline
    \rowcolor{blue!10}
    \textbf{Categoría Global} & \textbf{Detalle de Actividades Agrupadas} & \textbf{Total Casos} & \textbf{Porcentaje (\%)} \\
    \hline

    % =============================================================================
    % CONTENIDO DE LA TABLA 
    % =============================================================================
    \textbf{Tratamientos} & Enfermedades Bacterianas, Virales, Parasitarias, Metabólicas, Traumatismos, Intoxicaciones, Vitaminas/Minerales. & 288 & 45.71\% \\
    \hline

    \textbf{Inmunizaciones (Vacunaciones)} & Vacunación Parvovirus, Heptavalente, Triple felina, Rabia. & 139 & 22.06\% \\
    \hline

    \textbf{Desparasitaciones} & Desparasitación interna, externa, interna y externa. & 133 & 21.11\% \\
    \hline

    \textbf{Imagenología y Servicios Auxiliares} & Rayos X, Ecografía e Internación de pacientes. & 37 & 5.87\% \\
    \hline

    \textbf{Cirugías} & OVH (Canina y Felina) y Orquiectomia (Canina y Felina). & 33 & 5.24\% \\
    \hline

    \hline
    \rowcolor{green!10} % Verde claro para el total
    \textbf{TOTAL GENERAL} & Suma de todas las actividades reportadas en el internado & \textbf{630} & \textbf{100.00\%} \\
    \hline
\end{longtable}
Esto refleja la aplicación práctica de los conocimientos adquiridos durante el internado, abarcando desde la medicina preventiva (Inmunizaciones, Desparasitaciones) y las habilidades técnicas (Cirugías) hasta el manejo de patologías específicas (Tratamientos).
\end{landscape}

%\clearpage
%\begin{center}\rule{0.5\linewidth}{0.5pt}\end{center}
%\vspace*{-1cm}
\paragraph*{Análisis por Categoría Global:}

\leftenumerate{
\item \textbf{Tratamientos (45.71\%):} Esta categoría representa la mayor parte del trabajo clínico y abarca la aplicación de tratamientos terapéuticos adecuados a los animales, de acuerdo a las enfermedades diagnosticadas, incluyendo el seguimiento de la evolución.\\ Se destaca que la aplicación de vitaminas y minerales (125 casos) y el manejo de enfermedades bacterianas (69 casos, como Gastroenteritis y Traqueo-bronquitis) y virales (27 casos, como Parvovirus y Distemper) son componentes esenciales del manejo clínico diario.
\item \textbf{Inmunizaciones (22.06\%):} Las actividades de vacunación (139 casos) reflejan el objetivo de concientizar a la población sobre la necesidad de vacunar a sus mascotas contra las enfermedades presentes en la zona. La vacunación contra el Parvovirus Canino fue la actividad individual más frecuente dentro de este rubro (97 casos).
\item \textbf{Desparasitaciones (21.11\%):} Con 133 casos reportados, esta actividad, que incluye la desparasitación interna y externa, confirma un enfoque fuerte en la profilaxis y el control parasitario, elementos clave del control y profilaxis en la práctica veterinaria.
\item \textbf{Imagenología y Servicios Auxiliares (5.87\%):} Incluye la utilización de Rayos X y Ecografía (33 casos en total) para el diagnóstico. Estos métodos complementarios son fundamentales para la comprobación o negación de la hipótesis diagnóstica y para apoyar la toma de decisiones clínicas. El rubro también incluye la internación (4 casos).
\item \textbf{Cirugías (5.24\%):} Las intervenciones quirúrgicas (33 casos de OVH y Orquiectomía) demuestran el desarrollo de destrezas y habilidades en la práctica quirúrgica y postoperatoria, cumpliendo con uno de los objetivos específicos del internado. Estos procedimientos se realizaron siguiendo el método quirúrgico y protocolos técnicos.
}