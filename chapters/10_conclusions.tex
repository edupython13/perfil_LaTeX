\clearpage

\section{CONCLUSIONES}
    %Después de realizar el trabajo de internado en el municipio de Villa Vaca Guzmán, durante 6 meses.\\
    %Se llegó a obtener experiencia en el área de diagnósticos, tratamiento, prevención de enfermedades y cirugías.
    El periodo de internado en el Hospital para Animales, brindó una valiosa experiencia práctica y permitió la aplicación de los conocimientos teóricos adquiridos durante la carrera de Medicina Veterinaria y Zootecnia.
    
    %Se resalta la importancia de continuar actualizando los conocimientos y habilidades para brindar una atención de calidad a los pacientes.
    De acuerdo a los objetivos y resultados obtenidos durante el presente trabajo de internado en el Hospital para Animales, se llega a las siguientes conclusiones:
    
    \textbf{Fortalecimiento del Conocimiento Teórico-Práctico}: Se logró consolidar y fortalecer los conocimientos teóricos adquiridos durante la formación académica, aplicando métodos clínicos y pruebas complementarias para el diagnóstico eficaz y el tratamiento de las diferentes enfermedades que se presentaron.
    
    \textbf{Adquisición de Destreza Quirúrgica y Clínica}: Se desarrolló destreza y habilidad suficiente en la práctica quirúrgica y postoperatoria, ejecutando intervenciones y atenciones complejas (como OVH y Orquiectomías) que permitieron adquirir la experiencia necesaria para el manejo integral de los pacientes.

    \textbf{Manejo de Patologías Endémicas}: Se adquirió experiencia en el diagnóstico y tratamiento de las enfermedades más comunes observadas en el Hospital para Animales de Villa Vaca Guzmán. Esto incluye el tratamiento de enfermedades virales de alta prevalencia (como el Parvovirus Canino) y patologías parasitarias recurrentes (como la Sarna Sarcóptica), demostrando la aplicación de tratamientos terapéuticos adecuados.

    \textbf{Dominio Farmacológico y Terapéutico}: Se profundizó el conocimiento en la aplicación, dosificación y clasificación de fármacos y biológicos utilizados en el tratamiento y la prevención de enfermedades infecciosas y parasitarias, cumpliendo el objetivo de aplicar y profundizar el conocimiento terapéutico.

    \textbf{Cumplimiento de Objetivos Zoosanitarios}: Se lograron adquirir metodologías aplicadas al manejo zoosanitario, lo cual permitió cumplir con el objetivo de concientizar a la población sobre la necesidad de vacunar a sus mascotas contra enfermedades infectocontagiosas que prevalecen en la zona.
    
    %El internado permitió obtener experiencias y un criterio profesional propio indispensable para ejercer la profesión de médico veterinario.