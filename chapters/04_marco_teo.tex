\clearpage

\section{MARCO TEÓRICO}
    %--------------------------------------------
    \subsection{Farmacología}
        Se dice que la farmacología, es una ciencia que estudia las drogas y específicamente las reacciones que pueda tener al administrarlas dentro de un organismo vivo. \xcite{merck_co}
        
        \subsubsection{Objetivos de la Farmacología}
            La farmacología veterinaria por lo general apunta a recomendar, prescribir medicamentos terapéuticos para la clínica veterinaria. Para la prevención y el tratamiento de enfermedades que afectan a los animales en general. La manera en que actúa el fármaco contra el agente causal de la enfermedad y saber como debe aplicarse los medicamentos en cada especie.\\
            Las precauciones que deben tomar al usar cada fármaco, que antagonismo y sinergismo pueden producir dentro del organismo cuando dos o más medicamentos actúan de manera simultánea, las contra indicaciones y efectos colaterales que pueden causar cuando se emplea como agente terapéutico. \xcite{merck_co}
    %--------------------------------------------
    \subsection{Clínica Veterinaria}
        Se entiende por clínica veterinaria, al estudio de las manifestaciones morbosas de cualquier enfermedad que se observa en el animal a simple vista, hasta nos permite diagnosticar a través del historial clínico. \xcite{alexander1988}%(2 de anibal)
        
        \subsubsection{Objetivos de la Clínica Veterinaria}
            Nos permite diagnosticar presuntivamente todas aquellas enfermedades de los animales, en base a su historial clínico o anamnesis, proporcionando al médico veterinario datos necesarios acerca del curso de la enfermedad y las circunstancias que acompañan al observar a un paciente. \xcite{grov_cli}\\
            Importancia de la historia clínica, debido a que los animales no pueden describir signos y síntomas de una determinada enfermedad, además que una enfermedad tiene variación en un individuo de la misma especie, tomando en cuenta edad, raza, sexo, etc. \xcite{grov_pat}
    %--------------------------------------------
    \subsection{Conceptos Generales}
        
        \subsubsection{Patología}
            Es una rama que estudia las enfermedades y los trastornos que se producen en el organismo ocasionado por un agente etiológico. \xcite{geoffrey1993}%(6 sucre)
        
        \subsubsection{Síntomas}
            Son las manifestaciones de una alteración que sufre el organismo que puede ser identificado por el médico o el enfermo. \xcite{grov_cli}%(2 sucre)
        
        \subsubsection{Diagnóstico}
            Es parte de la medicina que tiene como principal objetivo la constatación e identificación de la enfermedad, representando la manifestación fundamental de la práctica médica, tomando como base la sintomatología de la misma. \xcite{grov_pat}%(3 sucre)
        
        \subsubsection{Tratamiento}
            Es un conjunto de medios de cualquier clase, farmacológico, higiénico, quirúrgico o físico, cuya finalidad es la curación o alivio de la enfermedad o síntoma, cuando se ha llegado a un diagnóstico. \xcite{geoffrey1993}%(6 sucre)
        
        \subsubsection{Examen Clínico}
            Para identifica cualquier enfermedad, depende del examen clínico que realiza el médico veterinario, pero esto también depende de la experiencia de la que es poseedor, porque en algunos casos puede llegar a un diagnóstico errado.\\
            Existen dos formas o métodos para realizar un examen clínico. \xcite{grov_pat}%(3 sucre)
            \paragraph{Primarios:} Clásicamente la exploración física es la percepción de los signos clínicos presentes en el paciente por los sentidos del médico, que se componen de: inspección visual, palpación, percusión, auscultación y olfacción.\par
            \paragraph{Secundarios:} Son maniobras sencillas realizadas por el médico con el apoyo de instrumentos como una fuente de luz, fonendoscopio, martillo de reflejos, estetoscopio, etc.
        
        \subsubsection{Técnica Quirúrgica}
            La técnica quirúrgica, estudia los procedimientos manuales e instrumentales, a través de los cuales los tejidos vivos son incididos y reconstruidos según un ordenamiento preconcebido con fines estéticos, económicos y de preparación para la terapéutica quirúrgica. \xcite{heidrich1974}\\%(7 sucre)
            Además coadyuva en la conservación, mejoramiento, reproducción y mayor producción de las especies animales útiles al hombre. La cirugía como uno de los pilares básicos de la medicina aplicada al hombre tiene su importancia en la medicina aplicada a los animales. \xcite{josesantos2004}%(8 sucre)
        
        \subsubsection{Anestesia}
            La anestesia general puede definirse como la supresión total, en forma temporal de la sensibilidad y de la movilidad de los seres vivos, sin afectar sus funciones vitales, mediante la acción de los fármacos aplicados por el médico veterinario. \xcite{alexander1988}%(1 sucre)
        
        \subsubsection{Fluido-terapia}
            La fluido-terapia, es la administración parenteral de líquidos y electrolitos, con el objeto de mantener o restablecer las homeostasis corporal, su fórmula es: $\%desh*kg~pv*10$. \xcite{alexander1988}%(1 sucre)
        
        \subsubsection{Tranquilizante}
            Es aquel fármaco que produce quietud o tranquilidad sin pérdida de la conciencia y sin tendencia al sueño. \xcite{alexander1988}%(1 subre)
    
        
        \subsubsection{Control y Profilaxis}
            Son acciones de salud que tienen como objetivo prevenir la aparición de una enfermedad o estado anormal del organismo, basado en un conjunto de acciones y consejos médicos. \xcite{geoffrey1993}
    %--------------------------------------------
    \subsection{Vacunación}
        La vacunación consiste en la inmunización a través de una inyección con el agente infeccioso o patógeno atenuado, ésta atenuación se obtiene por envejecimiento de los cultivos de patógenos por calentamiento, desecación o por antiséptico. \xcite{geoffrey1993}%(6 anibal)
    %--------------------------------------------
    \subsection{Enfermedades en Animales Menores y Mayores}
        
        \subsubsection{Parvo Virus Canino}
            \paragraph{Sinonimia:} Llamado también enteritis hemorrágica de los canes.
            \paragraph{Etiología:} Virus de las especie parvo virus canino.
            \paragraph{Definición:} Es una enfermedad muy aguda, ocasionada por el agente parvo virus, se presenta con mayor frecuencia en cachorros, esto se debe a la propia estructura del virus, que necesita para su duplicación tejidos que contengan un gran número de células en multiplicación activa. Esto explica la afinidad del virus hacia embriones, tejido linfoide, miocardio en las crías jóvenes y por el epitelio intestinal, los cachorros destetados registran un elevado grado de renovación en sus células.
            \paragraph{Síntoma:} Diarrea líquida sanguinolenta de olor fétido y vómito que produce una deshidratación, abdomen doloroso, anorexia, polidipsia, con temperatura normal pero tiende a disminuir cuando se complica. La piel de estos animales, pierde elasticidad, el pelo es opaco, postración de 1\-2 días después sí el animal no es tratado inmediatamente, ocurre la muerte por lesión cardiaca.
            \paragraph{Lesiones Macroscópicas:} En el tubo digestivo, enteritis hemorrágica, con contenido fecal oscuro sanguinolento, los ganglios linfáticos aumentados, en el hígado y el bazo hay distrofia tóxica.
            \paragraph{Lesiones Microscópicas:} A nivel del parénquima intestinal, se observa enteritis hemorrágica, necrosis en toda la capa epitelial con inclusión de células de las criptas de vellosidades que se encuentran un poco afectados, los folículos linfoides placas de seller, se encuentran con pequeña infiltración de linfocitos.
            \paragraph{Diagnóstico:} Está basado en una historia adecuada y en los signos clínicos, se confirma por prueba de elisa fecal o prueba de hemo-aglutinación positiva.\par
            La diarrea producida por el parvo virus, puede confundirse con otras enfermedades de curso similar, producidas por otros virus, bacterias o ambos, por ello es necesario realizar un examen viro-lógico para obtener un diagnóstico definitivo.
            \paragraph{Tratamiento:} Utilizar soluciones electrolíticas orales en perros ligeramente deshidratados y sin antecedentes de vómitos, los perros en condiciones graves deben recibir fluido terapia intravenosa (ringer lactato y dextrosa al 5\!\% con cloruro de potasio).\par
            Los vómitos se controlan con metoclopramida $0,2-0,5~mg/kg$ cuatro veces al día, por vía oral o subcutánea. En casos más graves, los que sufren de pérdida de sangre, fiebre o pérdida de la integridad intestinal; en éstos casos se debe administrar trimetropin sulfa $15~mg/kg$ dos veces al día, por vía subcutánea u oral durante 5-10 días, en casos más graves se recomienda usar ampicilina $20~mg/kg$ tres veces al día vía intravenosa y gentamicina $2,2~mg/kg$ tres veces al día, vía subcutánea durante un máximo de 5 días.
            \paragraph{Prevención:} Limpiar a fondo las zonas contaminadas, la desinfección de manos, ropa y recipientes para la comida y el agua. Los cachorros deben estar aislados de los perros adultos que regresan de exposiciones. Vacunar a los cachorros tres dosis a las 6, 9 y 12 semanas. \xcite{morgan1999}%(8 anibal)
        
        \subsubsection{Moquillo Canino}
            \paragraph{Sinonimia:} Distemper canino, llamado también hiperqueratosis de las almohadillas plantares o enfermedad del caminar rígido.
            \paragraph{Agente Etiológico:} Virus de la familias \emph{paramixoviridae}, la misma que provoca el sarampión en las personas.
            \paragraph{Definición:} Es una enfermedad importante y generalmente mortal, no solo para la especie canina, sino para varias especies de carnívoros domésticos, producida por un virus de fácil dispersión en el aire a través de aerosoles o gotas y a través de objetos contaminados, se contagia de una forma muy similar al virus de la gripe de los humanos.\par
            Esta enfermedad afecta principalmente al sistema nervioso y la médula espinal en el término de 10 a 14 días.
            \paragraph{Signos Clínicos:} Varían desde problemas respiratorios leves, como ojos y nariz aguados y con moco, diarreas severas, vómitos y ataques convulsivos.\par
            Cuando el virus ataca al sistema nervioso central (cerebro y médula espinal), pueden aparecer signo gravísimos que llevan al animal a una muerte casi segura, animales que hayan enfermado pueden quedar con espasmos musculares incontrolables de los miembros anteriores y posteriores, o convulsiones de aparición periódica como epilepsia.
            \paragraph{Lesiones Macroscópicas:} Hiperqueratosis de la nariz y en las almohadillas plantares de las pata. Dependiendo del grado de la infección bacteriana secundaria, también puede desarrollarse una enteritis o pústulas cutáneas, también la pared intestinal aparece engrosada, con alteraciones en el color y con desprendimiento de mucosa, puede haber material acuoso, oscuro o sanguinolento en estómago e intestino.
            \paragraph{Lesiones Microscópicas:} Necrosis en los tejidos linfáticos, neumonía intersticial, cuerpos de inclusión intra-citoplasmáticos e intra-nucleares en el epitelio respiratorio.
            \paragraph{Diagnóstico:} Cuadro febril con tos, secreciones óculo-nasal y signos digestivos en cachorros o adultos no vacunados, especialmente si va acompañado de cuadro inmunológico. También debe sospecharse en caso de aparecer cuadros nerviosos con o sin signos previos especialmente en perros viejos. \xcite{merck_co}%(11 sucre)
            \paragraph{Tratamiento:} Los pacientes con enfermedad respiratoria, deben recibir terapia con antibióticos por neumonía, con amoxicilina $20-30~mg/kg$ cada 8 horas vía oral o parenteral, también amoxicilina - ácido clavulanico $25~mg/kg$ cada 12 horas. Se aconseja añadir una terapia complementaria con vitaminas del grupo B, nebulizaciones, expectorantes y fisioterapia. \xcite{sodikoff1998}%(12 sucre)
            \paragraph{Prevención:} La única manera de evitar que un perro se contagie de distemper canino es a través de la vacunación. La edad adecuada para recibir la inoculación es entre las seis y las ocho semanas. Luego se le da un refuerzo anual de por vida.
        
        \subsubsection{Rabia}
            \paragraph{Sinonimia:} Llamada también hidrofobia.
            \paragraph{Agente Etiológico:} Producido por un virus de la familia \emph{rhabdoviridae}, género lissavirus, especie virus de la rabia.
            \paragraph{Definición:} Es una enfermedad vírica, cuya importancia fundamental es la de ser zoonosis; transmisible al humano, se transmite principalmente por la mordedura de un animal infectado, a través de la saliva. El virus penetra el tejido nervioso, para luego migrar hasta el sistema nervioso central y las glándulas salivales donde se libera, este sin embargo no resiste el calor, además que muchos desinfectantes lo inactiva fácilmente.
            \paragraph{Signos Clínicos:} Se manifiesta de dos a ochos semanas después de la infección, que es tiempo de incubación del virus. Esta enfermedad comprende tres fases:
                \subparagraph*{- Sin Signo Evidente:} La primera fase, pasa con frecuencia inadvertida, pero puede notar signos sutiles de cambio de comportamiento, fiebre, reflejos lentos y que el perro se lamen constantemente el sito de la mordida, como si tuviera mucha comezón.
                \subparagraph*{- Furiosa:} El sistema nervioso central esta invalidado, se notará signos de comportamiento errático, como irritabilidad, inquietud, ladridos, agresión por episodios, ataques a objetos inanimados, rascan, gruñidos inexplicables, fotofobia y comportamiento sexual anormal, también se desarrolla desorientación y convulsiones.
                \subparagraph*{- Paralítica:} Se desarrolla parálisis, que frecuentemente afecta a la extremidad mordida, luego la faringe (se percibe cambio de ladrido). Continúan problemas para respirar y parálisis de la mandíbula (caída), que provoca un exceso de salivación.
            \paragraph{Lesión Macroscópica:} Aumento del líquido céfalo-raquídeo, congestión, parálisis de la vejiga y del ano, estómago vacío y arrugado, hiperemia pasiva general.
            \paragraph{Lesión Microscópica:} En el cerebro y cerebelo, se observan corpúsculos de inclusión intra-citoplasmáticos eosinofilos a nivel de las neuronas.
            \paragraph{Diagnóstico:} Se hace un estudio de la cabeza y las glándulas salivales del animal, ya que cualquier perro sospechoso de rabia, se debe poner en cuarentena o someterse a eutanasia y las autoridades locales deben poner sobre aviso a la población del área.
            \paragraph{Tratamiento:} No hay tratamiento posible en animales, los humanos pueden sobrevivir si se vacunan rápidamente una vez que le ha mordido el animal.
            \paragraph{Prevención:} Vacunar anualmente a perros y gatos para conseguir un control de las enfermedad. \xcite{merck_co}%(10 anibal)
        
        \subsubsection{Hepatitis}
            \paragraph{Sinonimia:} Hepatitis canina contagiosa, hepatitis infecciosa, hepatitis vírica canina y endotelitis canina.\par
            Periodo de incubación, es de 6 a 9 días cuando los perros se ponen en contacto directo con otros.
            \paragraph{Agente Etiológico:} Adenovirus tipo 1, produce hepatitis en perros jóvenes no vacunados, el virus afecta las células del parénquima hepático y el endotelio vascular, provocando necrosis hepática.
            \paragraph{Síntomas:} Coloración amarilla en los ojos y la mucosa, pérdida de peso, fiebre, depresión, anorexia y dolor generalizado y deshidratación.
            \paragraph{Lesiones Macroscópicas:} Crecimiento de los ganglios linfáticos (amígdalas) en todo el cuerpo e hiperhemicos, frecuentemente presentan hemorragias en varios órganos y tejidos del cuerpo, como cerebro y médula espinal, meninges, timo, corazón, estómago y el intestino.
            \paragraph{Lesiones Mircroscópicas:} En el hígado, hiperemia, edema, hemorragia y necrosis coagulativa con una inflamación de neutrófilos, linfocitos y macrófagos, cuerpo de inclusión intra-nuclear, los cuales pueden ser basófilos o neutrófilos y están presente en el epitelio hepático y endotelio.
            \paragraph{Diagnóstico:} Hay tendencia hacia el edema en los vasos sanguíneos del cerebro, corazón, vesícula biliar y tejidos subcutáneos.\par
            El cuadro sanguíneo caracterizado por una neutrofilia pasajera, seguida de una leucopenia de desarrollo rápido.
            \paragraph{Tratamiento:} Terapia inmunosupresora para resolver o controlar el proceso inflamatorio; la terapia antioxidante para prevenir el estrés oxidativo, donde los antioxidantes se recomiendan como complemento a la terapia estándar para reducir la lesión hepática y la fibrosis en los perros, algunos de los cuales pueden incluir la vitamina E, la silibinina y el ácido ursodesoxicólico; y finalmente la terapia antifibrótica para inhibir claramente la fibrosis.
            \paragraph{Prevención:} El método preventivo más utilizado e importante contra la hepatitis canina es la vacunación obligatoria. Por lo general, su perro obtiene esta vacuna en conjunto con la del moquillo entre las 6 y las 8 semanas de edad.
        
        \subsubsection{Constipación}
            El estreñimiento puede darse a causa de la deshidratación, falta de fibra, obstrucción por cuerpo extraño, agrandamiento de la próstata, glándulas perianales bloqueadas o infectadas, tumores cerca de la región pélvica, hernia, falta de ejercicio.
            \paragraph{Síntomas:} Abdomen tenso o con hinchazón, heces duras, secas y pequeñas, el perro no evacua, esfuerzo, falta de apetito, postura encorvada, restos de heces en la zonal anal.
            \paragraph{Tratamiento:} Mucha agua, ejercicio, selección adecuada de los alimentos, mantener un horario de alimentación.
        
        \subsubsection{Gastroenteritis Hemorrágica Canina}
            \paragraph{Etiología:} Respuesta anafiláctica frente a endotoxinas bacterianas, también puede estar ocasionada por una enterotoxemia por clostridios.
            \paragraph{Signos:} Diarrea con sangre, retraso en el llenado capilar con pulso débil y rápido, la presencia de sangre digerida en las heces genera un olor característico similar al de la infección por parvovirus, depresión del sistema nervioso central, postración y coma.
            \paragraph{Diagnóstico Diferencial:} Infección por parvovirus, intoxicación por warfarina, colitis grave, ulceración gástrica, trombocitopenia, cuerpos extraños en el intestino.
            \paragraph{Lesiones Macroscópicas:} En los intestinos, se presentan áreas con hiperemia.
            \paragraph{Tratamiento:} El tratamiento empieza con fluido-terapia intensiva, para tratar la coagulación intravascular y la insuficiencia renal. A menudo se emplean antibióticos por vía parenteral como ampicilina y cloranfenicol. \xcite{gastro-jb}
        
        \subsubsection{Sarna Canina}
            Es una infección dermatológica, enfermedad de la piel causada por ácaros (arácnidos microscópicos).
            \paragraph{Agente Etiológico:} Existen varios tipos de ácaros produciendo cada uno de ellos una sarna diferente:\vspace{-3mm}
            \begin{enumerate}[itemsep=-3mm]
                \item Sarna demodéctica.
                \item Sarna sarcóptica.
                \item Sarna otodéctica.
            \end{enumerate}
            \paragraph{Sintomatología:} Picor y ardores constantes en la piel, lo que hará que el perro se rasque incesantemente, se frote contra muebles, paredes y el suelo. Heridas y llagas, consecuencia del constante rascado. Enrojecimiento e inflamación de la piel. Des-escamando de la piel, sequedad y aparición de costras. Mal olor, alopecia y pérdida de apetito y peso. \xcite{sarna_2019}
            \paragraph{Tratamiento:} Consistirá en acaricidas, que pueden ser administrados por vía oral, tópica o por medio de inyecciones.
        
        \subsubsection{Tumor Venéreo Transmisible Canino (TVT)}
            \paragraph{Sinonimia:} Tumor de Sticker, condiloma canino canino, linfosarcoma transmisible, histiocitoma contagioso, granuloma venéreo, tumor transmisible de células reticulares e histiocitoma. \xcite{birchard1996manual}
            \paragraph{Etiología:} La etiología no es clara, algunos autores discuten una causa retroviral y otros como un tumor de linfocitos, histiocitos o de células reticulares. Otras etiologías que contribuyen a su desarrollo pueden incluir a virus oncogénicos (papilomavirus, herpesvirus y particularmente el retrovirus).
            \paragraph{Transmisión:} A través del coito, desde animales afectados a animales sanos; hembra y macho están propensos a lesiones genitales durante el coito, por lo que son susceptibles al trasplante de células tumorales, también se presentan implantaciones extra genitales por mordedura, rascado, lamido u olfateo directo de la zona del tumor. Son los perros callejeros los que sirven como reservorios de ésta enfermedad.
            \paragraph{Signos Clínicos:} En el macho; hemorragias o pérdidas sanguinolentas por el pene en forma continua, también disuria ya que que la tumoración ocluye la uretra peniana. En la hembra; carnosidad que aparece en el orificio vulvar o pérdidas hemorrágica o sanguinolentas, que habitualmente cuando la tumoración no es visible son confundidas con un estro normal, que dura más tiempo del habitual.\par
            El tumor venéreo transmisible tiene dos formas de presentación:
            \begin{description}[nosep]
                \setlength{\itemsep}{-1mm}
                \item \hspace{-2mm}\textbf{Genital:}
                \begin{description}[nosep]
                    \setlength{\itemsep}{-1mm}
                    \item[Machos:] Desde el fornix hasta el prepucio peniano, debe exteriorizarse el pene para poder visualizarlo.
                    \item[Hembras:] En el vestíbulo vaginal, para poder visualizarlo se debe distender los labios vulvares.
                \end{description}
                \item[Extra-genital:] Las formas de presentación incluyen:
                \begin{itemize}[nosep]
                    \setlength{\itemsep}{-1mm}
                    \item Intra-nasal.
                    \item Cavidad oral (labios y lengua).
                    \item Ojo (Esclerótica y cámara anterior del ojo).
                    \item Piel.
                \end{itemize}
            \end{description}
            \paragraph{Diagnóstico.}
                \subparagraph{Examen Físico:} Descarga vulvar o prepucial hemorrágica. En los machos; las lesiones usualmente se localizan cranealmente en el pene, mucosa prepucial o glande. En las hembras; el tumor tiene macroscópicamente un aspecto similar y puede localizarse en el vestíbulo o canal vaginal protruyendo a través de los labios vulvares causando deformación de la región perianal.
                \subparagraph{Citología:} Se toma una muestra por raspados, aspirados o se realiza una impronta sobre el tumor. Las tinciones de Papanicolaou, seguida de Diff Quick son las mejores para diferenciar células tumorales.
                \subparagraph{Histopatología:} Microscópicamente este tumor se caracteriza por masas o láminas compactas de células neoplásicas que con frecuencia se disponen difusamente en hileras o racimos sobre un delicado estroma de tejido fibroso vascularizado. Las células son uniformes excepto por las formas celular atípicas y principalmente redondas, ovoides o poliédricas, pudiendo mostrar un intenso grado de necrosis.
            \paragraph{Diagnóstico Diferencial:} Debe diferenciarse de otras tumoraciones que afectan los genitales externos. En la vagina se observan leiomiomas o tumores epiteliales malignos y en la mucosa peniana pueden presentarse también tumores epiteliales malignos.\linebreak
            En las presentaciones extragenitales es necesario diferenciarlos principalmente del histiocitoma, mastocitoma o linfosarcoma.\par
            Un examen clínico cuidadoso, la revisión de los antecedentes y los resultados obtenidos de la citopatología o histopatología guían hacia un diagnóstico correcto.
            \paragraph{Tratamiento:} El tumor es curable en un porcentaje mayor al 95\% de los casos.
                \subparagraph{Cirugía:} Ampliamente usada para el tratamiento de pequeños tumores localizados, pero en casos de tumores grandes e invasivos el riesgo de contaminación con células de \emph{TVT} es muy alta.
                \subparagraph{Quimioterapia:} Es la terapia más eficaz y práctica, con el sulfato de vincristina como el fármaco usado con mayor frecuencia. La vincristina, se administra por vía endovenosa semanalmente en dosis de \(0.5 \; \text{a} \; 0.7 \: mg/m^2\) de superficie corporal o de \(0.025 \: mg/kg\) de peso vivo. La involución de las lesiones es muy notoria al inicio del tratamiento, la remisión completa generalmente toma de 2 a 8 inyecciones. El tratamiento es exitoso en fases iniciales de la enfermedad, concretamente en casos menor a un año de duración. En casos de mayor duración, se requieren periodos más prolongados de terapia y la tasa de curación es menor.
            \paragraph{Prevención:} Evaluaciones periódicas con un médico veterinario y si se observan zonas sospechosas, realizar improntas con un portaobjetos. En hembras evaluar la mucosa vulvar y la vagina, tomar muestras del flujo vaginal.
        
        \subsubsection{Carbúnculo Hemático}
            \paragraph{Sinonimia:} Ántrax.\par
            Es una enfermedad contagiosa, sumamente drástica, que provoca muerte de los animales de forma súbita, por lo que se le llama enfermedad del rayo.
            \paragraph{Agente Etiológico:} Es una bacteria esporulada Bacillus anthracis, que a diferencia de los clostridium es aerobia. También produce esporas, las cuales pueden sobrevivir en el medio ambiente varios años.
            \paragraph{Patogenia:} La principal fuente de infección son los animales muertos y las esporas que contaminan praderas, campos y fómites.\par
            En contacto con el oxígeno, la bacteria produce esporas que son resistentes y que sobreviven durante años en el suelo o en la lana o el pelo de los animales infectados. Las esporas pueden penetrar en el cuerpo de un animal por ingestión o inhalación o a través de heridas en la piel, allí germinan y causan la enfermedad. Como la sangre de los animales infectados no siempre se coagula correctamente, el animal puede sangrar a través de los orificios corporales y los insectos transmitirán la bacteria a otros animales. Los carnívoros y el ser humano pueden adquirir la infección si consumen la carne de un animal infectado. Sin embargo, la infección de los animales se produce en general por la ingestión de esporas que se encuentran en el suelo o en los piensos.
            \paragraph{Sintomatología:} El animal presenta fiebre, hemorragias por orificios naturales, severa depresión, hipoglucemia severa, shock séptico y muerte.
            \paragraph{Prevención:} La enfermedad se presenta al culminar la temporada de lluvia, por lo que se recomienda vacunar a los animales antes de que termine la temporada. La vacunación es anual, pero en zonas problema se realizará cada 6 meses. También los animales muertos deben ser incinerados. \xcite{josesantos2004}%(11 anibal)
        
        \subsubsection{Brucelosis Bovina}
        %    Es una enfermedad infecto contagiosa, causada por una bacteria del género brucella, que afecta a animales domésticos y al humano. En el bovino la especie responsable, es brucella abortus que se caracteriza por producir abortos en el último tercio de gestación y muerte en recién nacidos, retención de placenta e infertilidad.\par
        %    Los animales infectados, eliminan millones de bacterias semanas antes y posterior al parto o aborto, que constituye en fuente de contaminación.\par
        %    El diagnóstico puede realizarse a través del aislamiento bacteriológico del agente causal, como también por reacciones serológicas en suero o leche de animales sospechosos.\par
        %    Las medidas de prevención, consiste en eliminación de los animales infectados y control de los animales que ingresen al establecimiento. \xcite{josesantos2004}%(11 anibal)
            La brucelosis bovina es una enfermedad infecto-contagiosa de distribución mundial que afecta en forma primaria a los bovinos, pudiendo transmitirse a los humanos y a otras especies animales.
            \paragraph{Sinonimia:} Aborto bovino, aborto infeccioso o contagioso, fiebre ondulante, enfermedad de Bang y en humanos fiebre de Malta.
            \paragraph{Etiología:} El agente causal es la bacteria Brucella Abortus, un cocobacilo o bacilo corto, aeróbico, gram negativo. Éste microorganismo es un patógeno intracelular facultativo. La enfermedad clínica ocurre durante la madurez sexual y se manifiesta por abortos, nacimientos prematuros, baja fertilidad, retención placentaria, infecciones uterinas, epididimitis, orquitis e infertilidad en machos. La presentación del cuadro clínico puede ser aguda o crónica.
            \paragraph{Patogenia:} Las especies del género brucella, son patógenos intracelulares facultativos, propiedad que las mantiene protegidas de la acción de los antibióticos y de los anticuerpos del sistema inmune, lo que le confiere una naturaleza crónica a la enfermedad.\par
            Cuando las bacterias ingresan al organismo, son atacadas por células del sistema inmune que las destruyen en su interior. Si no son destruidas se desplazan por los conductos linfáticos y llegan a los ganglios, donde son distribuidas por vía sanguínea al resto del organismo, teniendo tropismo especialmente por la placenta ya que ésta produce algunos metabolitos esenciales para el desarrollo de la bacteria en éste órgano. \xcite{merck_co}
            \paragraph{Especie Susceptible:} Bovinos hembras y machos a partir de los 12 meses de edad.
            \paragraph{Fuentes de Infección:} Por contacto con fetos abortados, membranas y fluidos placentarios, secreciones vaginales de animales infectados, transplacentaria y por vía digestiva a través de leche, agua, pasto contaminado.
            \paragraph{Periodo de Incubación:} Es variable con un rango de 10 días a 7 meses.
            \paragraph{Morbilidad y Mortalidad:} En los animales sin vacunación ni exposición a la brucella abortus, se propaga rápidamente con una tasa de abortos que oscila entre 30\% y 80\%. En los predios o rodeos donde el organismo se ha vuelto endémico, solo aparecen síntomas esporádicos y las vacas pueden abortar durante su primera preñez.\par
            Debido a la resistencia genética del ganado bovino, las muertes en animales adultos suelen ser escasas.
            \paragraph{Signos Clínicos:} Abortos y mortinatos; los abortos suelen ocurrir durante el último tercio de gestación, algunos terneros nacen débiles y pueden morir poco tiempo después de nacer. También puede producir retención de placenta y metritis secundaria, algunas veces se observan epididimitis, vesiculistis seminal, orquitis o abscesos testiculares en los toros. La infertilidad ocurre en ambos sexos debido a la metritis o a la orquitis, epididimitis. \xcite{castro2005brucelosis}
            \paragraph{Diagnósticos:}
                \subparagraph{Clínico:} Aborto en el último tercio de la gestación y también el nacimiento de terneros débiles.
                \subparagraph{Diferencial:} Se deben tomar en cuenta otras enfermedades que causan abortos o epididimitis y orquitis en el ganado bovino, el diagnóstico diferencial incluye tricomoniasis, leptospirosis, rinotraqueitis infecciosa bovina.
                \subparagraph{Laboratorio:}
                    \begin{itemize}[nosep]
                        \setlength{\itemsep}{-1mm}
                        \item Screening o Tamizaje:\vspace{-1mm}
                        \begin{enumerate}[noitemsep]
                            \setlength{\itemsep}{-1mm}
                            \item Leche: ELISA indirecto (ELISA-I).
                            \item Suero: Rosa de Bengala (RB), Fluorescencia Polarizada (FP).
                        \end{enumerate}
                        \item Confirmación:\vspace{-1mm}
                        \begin{enumerate}[noitemsep]
                            \setlength{\itemsep}{-1mm}
                            \item Suero: Elisa competitivo (C-ELISA).
                            \item Ganglios, Tejidos, Órganos, Leche: Cultivo bacteriológico.
                        \end{enumerate}
                    \end{itemize}
            \paragraph{Tratamiento:} La terapia antibiótica, no se recomienda debido al riesgo zoonótico, costo y efectividad debido a que la localización de la brucella es intra-celular, para el tratamiento se requiere la asociación de más de un anti-microbiano por varias semanas, lo que resulta costoso a la hora de evaluar costo beneficio.
            \paragraph{Prevención y Control:} Para la prevención de la brucelosis bovina, se utiliza como primera medida, la vacunación de terneras entre 3 y 8 meses de edad con vacunas que contengan Cepa-19 y RB-51.\par
            Como Medida de control, sólo deben ingresar al hato animales no infectados provenientes de fincas libre de brucelosis y se deben eliminar a los animales infectados y a los hijos de vacas confirmadas como positivas.

        \subsubsection{Fiebre Aftosa}
            Es una enfermedad viral muy contagiosa, de curso agudo con de alta mortalidad y morbilidad. Se caracteriza por lesiones y erosiones en el epitelio de la boca, fosas nasales, morro, patas, tetillas, ubre.
            \paragraph{Etiología:} Causado por un entero-virus aphotovirus, de la familia picornaviridae, sensible a cambios en el pH, exposición a luz solar y altas temperaturas. Es resistente al éter, cloroformo y desinfectantes como el hidróxido de sodio, corbato de sodio y ácido acético.\par
            El virus es difundido por el viento y la principal fuente de contagio es por medio de aerosoles a través de las vías respiratorias, el virus se aloja y multiplica en las células de la garganta, luego pasa al sistema circulatorio e infectar órganos.
            \paragraph{Síntomas:} Apatía, falta de apetito, fiebre y escalofrío, chasquidos de los labios, temblores de las patas, vesículas en las fosas nasales, cavidad bucal y uñas, mal estado general.
            \paragraph{Diagnóstico:} Signos clínicos semejantes a la estomatitis vesicular, por lo que se recomienda realizar pruebas de laboratorio como fijación de complemento, neutralización del virus, precipitación en agar-gel y elisa.
            \paragraph{Tratamiento:} No hay tratamiento específico, por lo que los animales infectados deben someterse a sacrificio e incineración. Aplicar vacuna anti-aftosa, según plan de vacunación. \xcite{grov_cli}%(3 anibal)
        
        \subsubsection{Tuberculosis Bovina}
            Enfermedad infecciosa causada por bacilos patógenos resistentes al ácido, es una enfermedad progresiva de curso muy debilitante.
            \paragraph{Etiología:} Mycobacterium bovis, es una bacterias capaz de producir enfermedad en la mayoría de los animales de sangre caliente incluido el humano. Se localiza en los pulmones formando una masa granulosa parecida a un tumor para posteriormente formar masas gaseosas con tendencia a la mineralización.
            \paragraph{Síntomas:} Ganglios linfáticos superficiales agrandados, debilidad, anorexia, disnea, emaciación, fiebre fluctuante y tos seca.
            \paragraph{Diagnóstico:} Realizar prueba de tuberculina.
            \paragraph{Tratamiento:} No existe tratamiento específico y en la mayoría de los casos los animales infectados, sacrificados. \xcite{heidrich1974}%(5 anibal)
    %--------------------------------------------
    %\subsection{Parásitos Internos}
    %    Los parásitos internos son pequeños organismos, que viven y se alimentan dentro de los órganos del huésped, afectando su desarrollo.\par
    %    Estos parásitos al robar nutrientes, producen diarrea, pérdida de peso, anemia, pelo erizado y sin brillo, falta de apetito y debilidad.\par
    %    Para que los animales se vean libres de parásitos, se debe tener agua limpia, evitar que beban agua estancada y administrarle desparasitantes que los eliminen e inactiven los huevos que estos ponen.
    %--------------------------------------------
    \subsection{Cirugía en Animales Menores}
        \subsubsection{Cesárea en Canes}
            \paragraph{Materiales}
            
            \begin{center}
                %\vspace{0.5cm}
                \begin{minipage}{\linewidth}
                    \begin{multicols}{2}
                        \begin{itemize}[label={\textbullet},left=0pt,itemsep=0pt,topsep=0pt,parsep=0pt,partopsep=0pt]
                            \item Xilacina al 2\%.
                        \item Sulfato de atropina.
                        \item Ketamina (Clorhidrato de Ketamina).
                        \item Pentagal (Antibiótico, desinflamante).
                        \item Oxitec (Oxitetraciclina).
                        \item Hila catgut crómico.
                        \item Hilo de nylon externo.
                        \item Bisturí.
                        \item Gasas.
                        \item Torundas.
                        \end{itemize}
                    \end{multicols}
                \end{minipage}
            \end{center}

            \paragraph{Procedimiento:} Colocar la sábana abierta más las compresas, se limpia con yodo la piel y hace la incisión ligeramente oblicua dorso ventral de 8 a 12 cm, luego se introduce la mano en la cavidad abdominal y se ubica la bifurcación de los cuernos en su unión con el útero, para llevarla hacia la herida de la pared y exteriorizarla lo máximo para posteriormente hacer una herida en la bifurcación del tamaño que permita la salida de los fetos.\par
            Se hace hemostasia y se entregan a su ayudante, que quitará la envoltura y cortará el cordón umbilical entre dos pinzas hemostáticas y por la misma herida se extraen todos los fetos y verificando que no quede ningún feto en el útero.\par
            Cuando los fetos están vivos y la placenta permanece adherida al útero, no debe forzarse su extracción además que debe ser lenta, de lo contrario se corre riesgo de hemorragia.\par
            Finalizada con la extracción de los fetos, se trata la herida con suero fisiológico tibio y se coloca en el interior del útero dos bolos uterinos de oxitec, para iniciar con la sutura de cushing con el hilo crómico \#2, durante intervención un ayudante detendrá las paredes uterinas con compresas húmedas, al culminar se regresa el útero a la cavidad abdominal para luego suturar el peritoneo junto al músculo transverso en forma continua, se procede a limpiar la herida con agua oxigenada para aplicar pomada bactericida.\par
            La limpieza de la herida debe realizarse hasta el cuarto día y el tratamiento post operatorio con penicilina a fin de evitar el secado de la leche. \xcite{alexander1988}%(2 anibal)