\clearpage

\section{RECOMENDACIONES}
    %\begin{itemize}
    %    \item Invertir en la adquisición de equipos y reactivos de laboratorio para el Hospital Para Animales, ésta acción permitirá ofrecer un servicio de diagnóstico más completo y preciso, beneficiando a los pacientes como a los estudiantes.
    %    \item Implementar programas de educación para concienciar a la población sobre la importancia de la salud animal y la tenencia responsable de mascotas.
    %    \item Fomentar la investigación en el área de medicina veterinaria para el desarrollo e implementación de nuevas técnicas de diagnóstico y tratamiento.
    %\end{itemize}

    Al concluir este periodo de internado, y tras un análisis de las actividades realizadas en el Hospital para Animales, se plantean las siguientes recomendaciones para optimizar la calidad de la atención clínica, la cirugía y la medicina preventiva, tomando como referencia las necesidades observadas en mi trabajo y las experiencias de otros internos:

    \textbf{1. Fortalecimiento Diagnóstico y Mejora de la Precisión Clínica}
    
    Es fundamental que el Hospital para Animales fortalezca su capacidad diagnóstica, lo cual repercutirá directamente en la efectividad de los tratamientos aplicados (tal como se busca en los objetivos de la clínica).

    \textbf{Adquisición de Equipos y Reactivos}: Se recomienda la inversión en la adquisición de equipos y reactivos de laboratorio. Tal como se ha observado en el ejercicio de la clínica, la utilización de laboratorio es crucial para obtener mayor validez en diagnósticos, pronósticos y tratamientos. Una dotación completa permitirá ofrecer un servicio de diagnóstico más preciso y completo, beneficiando tanto a los pacientes como a la formación de los estudiantes.
    
    \textbf{Contratación de Personal Especializado}: Se debe considerar prioritaria la contratación de un laboratorista, con el fin de realizar pruebas de laboratorio en diversos aspectos para obtener resultados más precisos en las patologías diagnosticadas, facilitando así la instauración de tratamientos efectivos.
    
    \textbf{Adquisición de Equipos de Urgencia}: Es necesaria la gestión para la adquisición de equipos y ambulancia veterinaria para realizar campañas y asistencias técnicas oportunas hacia los productores en casos de emergencia, un punto que debe ser cubierto en una institución dependiente de la Facultad Integral Defensores del Chaco.
    
    \textbf{2. Educación Zoosanitaria y Orientación a la Población}

    Dados los altos índices de enfermedades prevenibles (como el Parvovirus Canino, que sigue siendo altamente endémico), se debe dar un énfasis continuo a la medicina preventiva y la concientización.

    \textbf{Orientación sobre Calendario de Vacunación}: Es crucial orientar a los dueños de mascotas sobre la importancia de llevar un calendario de vacunación y desparasitación, y cumplirlo, ya que esto beneficia tanto a la salud animal como a la salud humana, previniendo enfermedades zoonóticas.
    
    \textbf{Promoción de la Asistencia Veterinaria}: La Facultad Integral Defensores del Chaco debe concientizar a la población del municipio de Villa Vaca Guzmán sobre la existencia y los servicios del Hospital para Animales (laboratorio, cirugía, clínica y urgencias).
    
    \textbf{Cumplimiento Normativo}: Se debe promover el cumplimiento de todas las normas sanitarias y la educación zoosanitaria, con el fin de lograr el bienestar animal de acuerdo con la ley de protección animal.
    
    \textbf{3. Fortalecimiento Institucional y Desarrollo Profesional}

    Para garantizar una atención de alta calidad y mantener la relevancia del Hospital, se debe invertir en el desarrollo de su capital humano y la coordinación con el sector productivo.

    \textbf{Actualización y Especialización del Personal}: Se recomienda la actualización permanente del personal para brindar una atención especializada con alta calidad, garantizando un diagnóstico clínico y patológico superior.
    
    \textbf{Coordinación con Productores}: La dirección del Hospital para Animales debe coordinar con las asociaciones de productores pecuarios, realizando talleres de capacitación con asistencia técnica para el control de enfermedades infecciosas.
    
    \textbf{Fomento a la Investigación}: Es recomendable fomentar la investigación en el área de medicina veterinaria para el desarrollo e implementación de nuevas técnicas de diagnóstico y tratamiento, aprovechando el entorno académico del Hospital.