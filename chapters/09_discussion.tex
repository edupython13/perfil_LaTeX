\clearpage

\section{DISCUSIÓN}
    %Concluido el trabajo de internado en clínica de animales menores y mayores en el municipio de Villa Vaca Guzmán.
    %\begin{table}[!h]
    %    \renewcommand{\arraystretch}{2}
    %    \centering
    %    \begin{tabular}{|l|c|r|r|}
    %        \hline
    %        \textbf{Detalle} & \textbf{Cantidad} & \textbf{Precio(Bs)} & \textbf{Precio Total(Bs)} \\ \hline
    %        Alimentación & 7 meses & 350 & 2450 \\ \hline
    %        Ropa de trabajo & 2 & 300 & 600 \\ \hline
    %        Material de escritorio & Global &  & 300 \\ \hline
    %        Empastado & Global &  & 200 \\ \hline
    %        Imprevistos 10\% &  &  & 350 \\ \hline
    %        \textbf{Total} &  &  & 3900 \\ \hline
    %    \end{tabular}
    %    \caption{Elaboración, fuente propia.}
    %    %\label{tab:my-table}
    %\end{table}
    %
    Al finalizar el periodo de internado en el Hospital para Animales del municipio de Villa Vaca Guzmán, se consolidaron los conocimientos teórico-prácticos, logrando ejercer destrezas en el diagnóstico, tratamiento y manejo quirúrgico de las patologías más prevalentes en animales menores y mayores, como se establecieron en los objetivos del trabajo.
    
    Los resultados obtenidos en las diferentes áreas de atención se discuten a continuación, contrastados con la experiencia reportada por otros internos:
    %1. Incidencia de Enfermedades Virales y Patologías Endémicas
    \subsection{Incidencia de Enfermedades Virales y Patologías Endémicas}
    La atención de enfermedades virales caninas se reafirma como una actividad central en el Hospital para Animales.
    Durante este internado, se registraron 32 casos de enfermedades virales, siendo el Parvovirus Canino la patología más atendida (31 casos). Esta alta incidencia es consistente con la experiencia de la Univ. Alcira Rocha Soliz, quien también reportó una prevalencia alta de Parvovirus (31 casos), aunque su total de atenciones virales fue mayor (42 casos).
    Al comparar con el total de atenciones del Univ. Kevin García Martínez en la misma institución, quien atendió 25 casos de enfermedades víricas, los resultados de este internado (32 casos) sugieren que la enfermedad viral sigue fluctuando con alta intensidad en la zona.\\
    Se observa que el Parvovirus continúa siendo una enfermedad devastadora en cachorros, requiriendo un tratamiento intensivo. La Univ. Daniela Morales Padilla, en la Clínica Veterinaria Sudamericana, también destacó la alta prevalencia del Parvovirus, reportando 70 casos tratados, y describió la enfermedad como una gastroenteritis hemorrágica.
    Respecto a las intoxicaciones, se atendieron 13 casos de intoxicación alimenticia. Este valor es muy similar al reportado por Kevin García Martínez, quien documentó 15 casos de intoxicación (2.69\% del total de actividades). Esta similitud de cifras indica que los casos de urgencias toxicológicas mantienen una demanda constante y predecible en el Hospital para Animales de Villa Vaca Guzmán.
    %2. Desarrollo de Habilidades Quirúrgicas y Atención de Urgencias
    \subsection{Desarrollo de Habilidades Quirúrgicas y Atención de Urgencias}
    Uno de los objetivos específicos fue desarrollar destrezas y habilidades en la práctica quirúrgica y postoperatoria.\\
    Se realizaron un total de 26 cirugías mayores visibles en los cuadros principales (14 OVH, 12 Orquiectomías). Además, se incluyeron cirugías especializadas como la extirpación de Tumor Venéreo Transmisible (TVT) y extirpación de fibrosarcoma.\\
    El volumen quirúrgico obtenido (26 casos en cirugías mayores documentadas en los cuadros principales, más procedimientos menores) es significativamente mayor que las 15 cirugías reportadas por la Univ. Alcira Rocha Soliz, lo cual supuso una excelente oportunidad para mejorar las destrezas.\\
    Sin embargo, al compararse con el Univ. Kevin García Martínez en el mismo hospital, mi trabajo implicó un menor número de intervenciones, ya que Kevin reportó 72 cirugías (12.9\% del total de atenciones). Kevin ya había contrastado su alto número de cirugías (72) con informes anteriores (GUARACHI, 2015) que registraron 40 cirugías. Mi experiencia se situó en un punto medio, permitiendo la aplicación continua de técnicas quirúrgicas.
    %3. Actividades de Profilaxis y Concientización (Vacunación y Desparasitación)
    \subsection{Actividades de Profilaxis y Concientización (Vacunación y Desparasitación)}
    Las actividades de profilaxis son esenciales para el control de enfermedades zoonóticas.\\
    Se administraron un total de 139 inmunizaciones (97 Parvovirus, 36 Heptavalente, 5 Triple Felina y 1 Rabia). Este resultado es inferior a las 183 vacunaciones realizadas por Kevin García Martínez (32.80\%) en el mismo centro. Kevin ya había notado que sus cifras eran inferiores a las 392 vacunaciones reportadas por CEREZO (2015), lo que demuestra una variabilidad constante en el volumen de profilaxis realizado en el hospital.\\
    La baja cifra en el Hospital para Animales contrasta fuertemente con la experiencia reportada por Daniela Morales Padilla en la Clínica Veterinaria Sudamericana, quien registró 1150 inmunizaciones, lo que sugiere, como ella misma concluyó, una mayor conciencia por parte de los propietarios en esa área urbana para cumplir con los calendarios de vacunación. Esto refuerza la necesidad de continuar concientizando a la población en Villa Vaca Guzmán sobre la importancia de la vacunación.\\
    En cuanto al control parasitario, se realizaron 258 tratamientos profilácticos (incluyendo desparasitación interna y externa, Ivermectina, y aplicación de vitaminas/minerales). Este alto volumen, especialmente en desparasitaciones, se alinea con la experiencia de Kevin García Martínez, quien también documentó 148 desparasitaciones (26.56\%), superando la cifra de HEREDIA (9.82\%) en el mismo establecimiento. La consistencia de estos resultados subraya que el control de parásitos (como la sarna sarcóptica, que es prevalente) es una demanda constante y de alta relevancia en la clínica veterinaria de la región.\par
    En resumen, el internado me permitió obtener experiencia directa y fortalecer las habilidades de diagnóstico y tratamiento en patologías endémicas (como Parvovirus) y adquirir destrezas en la práctica quirúrgica y postoperatoria, logrando formar un criterio profesional propio.