\clearpage

\section{METODOLOGÍA}
    %En el trabajo de internado en el Hospital para animales de la Facultad Defensores del Chaco de la Universidad Mayor Real y Pontificia San Francisco Xavier de Chuquisaca, se aplicarán diferentes métodos según se presenten una amplia variedad de casos que serán atendidos.\par
	%\subsection{\normalsize{Método de Observación}}
    %    Éste método consiste en ver al animal, las mucosas oculares, nasal y la auscultación completa de los signos vitales.
	%    \subsubsection{\normalsize{Palpación}}
    %        Esta se realiza con los dedos y ahí sentimos las lesiones o anormalidades como tumores, edemas, hinchazón, etc.
	%    \subsubsection{\normalsize{Auscultaciones}}
    %        Mediante esta técnica, podemos escuchar los ruidos pulmonares, frecuencia cardiaca a través de un estetoscopio.
    		%\begin{enumerate}[nosep]
    		%	\item Identificación del paciente en la ficha clínica.
    		%	\item Con medidas antisépticas, se extraerán $2\text{ml}$ de sangre de la vena cefálica.
    		%	\item Cada muestra será homogeneizada con el anti-coagulante y refrigerada a \(4^\circ\text{C}\).
    		%	\item Cada muestra será rotulada con su respectiva información (datos del paciente).
    		%\end{enumerate}
	%    \subsubsection{\normalsize{Percusión}}
    %        Consiste en golpear el cuerpo de animal, para provocar sonidos característicos.
    %\subsection{\normalsize{Método Farmacológico}}
    %    Es la aplicación de los diferentes fármacos en base a las dosificaciones que se requieran para cada caso y el posterior tratamiento hasta la recuperación del paciente.
    %\subsection{\normalsize{Método Quirúrgico}}
    %    Es el apoyo de las distintas técnicas descritas, para cada proceso quirúrgico y operaciones como cirugías estéticas, cesáreas, castraciones, ligaduras, etc.
    %\subsection{\normalsize{Método Terapéutico}}
    %    Aplicar las distintas terapias para la variedad de casos que surjan durante el trabajo de internado.
    %\subsection{\normalsize{Método Epidemiológico}}
    %    Se aplicará en toda en área de influencia que pueda tener el hospital para animales, empleando programas de bio-seguridad y asistencia técnica veterinaria.
    %\subsection{\normalsize{Análisis Estadístico}}
    %    Se utilizará para la tabulación de datos en la realización del presente trabajo de internado, también para agrupar la información obtenida.
    %
    %\begin{center}\rule{0.5\linewidth}{0.5pt}\end{center}
    %    
    La metodología utilizada en el presente trabajo de internado se centra en la aplicación sistemática y rigurosa del método científico adaptado a la práctica clínica. Este capítulo describe la estrategia metodológica empleada para la atención de pacientes, la documentación de la casuística y el fortalecimiento de las destrezas profesionales.

    \subsection{Estrategia Metodológica}

    La estrategia metodológica principal se basa en el \textbf{Método Clínico}, el cual es fundamental para la búsqueda de un diagnóstico y el establecimiento de una terapéutica adecuada en pacientes individuales.

    El diseño de investigación de este trabajo se clasifica como:

    \leftitemize{
        
        \item \textbf{Según su Naturaleza:} Se utilizó un enfoque mixto o predominantemente \textbf{cuantitativo} y \textbf{cuantitativo descriptivo}, ya que se buscó recolectar datos numérico (como en el método estadístico) y obtener información detallada sobre las características de los fenómenos clínicos (casos y síntomas).
        
        \item \textbf{Según la Fuente:} Se trata de una \textbf{investigación de campo}, pues se realizó en el mismo lugar y tiempo donde ocurrió el fenómeno (la atención de los pacientes en clínica).
    }

    \subsection{Métodos Teóricos}

    Los métodos teóricos fueron empleados para el desarrollo del razonamiento clínico y la estructuración del proyecto.

    \subsubsection{Método Inductivo}

    Este método se basó en la observación y la experimentación de hechos y acciones concretas (los signos clínicos y la casuística del paciente). Permitió ascender de lo particular (síntomas y hallazgos) a lo general (diagnóstico o conclusión general). Este fue empleado en el análisis e interpretación de la información obtenida para considerar la casuística, la sintomatología, las consecuencias y la posible patología de afección del paciente.

    \subsubsection{Método Deductivo}

    Este método fue útil al momento de realizar la recepción del paciente, la elaboración de la ficha clínica, la valoración y la presunción de la patología para su tratamiento respectivo en la clínica. Se utilizó para deducir conclusiones lógicas y aplicar principios generales y protocolos estandarizados (la teoría) a cada caso específico.

    \subsection{Diagnóstico Semiológico (Método Clínico)}

    El método semiológico estudia los métodos del examen clínico, investiga los signos y síntomas, e indica sus mecanismos y valores para construir el diagnóstico y deducir el pronóstico.

    Las fases del método clínico aplicadas fueron:

    \leftenumerate{
        
        \item \textbf{Formulación del Problema:} Identificación clara y precisa del motivo principal que llevó al dueño a consultar al médico veterinario, que generalmente es el trastorno o pérdida de la salud del animal.
        
        \item \textbf{Obtención de la Información Primaria:} Desarrollo del examen clínico completo, constituido por la anamnesis (interrogatorio) y el examen físico.
        
        \item \textbf{Formulación de la Hipótesis Diagnóstica:} Planteamiento de las hipótesis posibles fundamentadas en la información obtenida del examen clínico.
        
        \item \textbf{Comprobación o Negación de la Hipótesis:} Realización de un protocolo adecuado y planificado de investigaciones (pruebas paraclínicas) para comprobar o rechazar la hipótesis.
    }

    \subsubsection{Examen Físico (Exploración Clínica)}

    El examen físico es la percepción de los signos clínicos presentes en el paciente por los sentidos del médico. Este procedimiento se desarrolló mediante las siguientes técnicas, consideradas primarias:

    \leftitemize{
        
        \item \textbf{Inspección:} Observación visual del paciente en su conjunto para evaluar su estado general.

        \item \textbf{Palpación:} Procedimiento que consistió en la aplicación del sentido del tacto. Se palpan las zonas que presumiblemente están afectadas para apreciar consistencia, elasticidad, tumores, temperatura.

        \item \textbf{Auscultación:} Método para escuchar los sonidos normales o patológicos producidos en los órganos del cuerpo (pulmones, corazón, intestinos, etc.) mediante un estetoscopio.

        \item \textbf{Percución:} Maniobra que consistió en dar golpes suaves en el cuerpo del paciente para generar un sonido que provoca vibración en los tejidos u órganos subyacentes. Ésta vibración produce un ruido de intensidad y tonos diferentes, siendo de interés para el examen clínico.

        \item \textbf{Olfación:} Técnica de exploración utilizada para determinar posibles alteraciones cuantitativas o cualitativas en la función olfatoria.

    }

    \subsection{Métodos Especializados de la Clínica}

    Estos métodos representan la aplicación de destrezas técnicas necesarias para el tratamiento y la intervención.

    \subsubsection{Método Farmacológico}

    Se aplicó este método para la prevención y el tratamiento de enfermedades, lo que implicó la selección de medicamentos terapéuticos específicos, la especificación de vacunas y antiparasitarios, y el conocimiento de dosis y vías de administración.

    \subsubsection{Método Quirúrgico}

    Este método fue desarrollado de acuerdo con las técnicas aprendidas durante la formación académica para realizar procedimientos manuales e instrumentales en el paciente. Implica la preparación del paciente antes del quirófano, la aplicación de anestésicos, el control durante la operación y la posterior recuperación.

    \subsubsection{Método Terapéutico}

    El método terapéutico se orientó a buscar la solución o el alivio del problema a partir de los síntomas y el diagnóstico. Se determinó el uso de un tratamiento que incluiría la dosis, la vía de administración adecuada y la duración del mismo para cada caso particular.

    \subsubsection{Métodos de Diagnósticos Complementarios}

    Los análisis complementarios confirman una hipótesis y asisten en el tratamiento de afecciones. Estos incluyen:

    \leftitemize{

        \item \textbf{Pruebas de Laboratorio:} Análisis de sangre (hemograma), orina y heces para detectar enfermedades metabólicas o infecciones.

        \item \textbf{Imagenología:} Uso de rayos X y ecografías para visualizar estructuras internas detalladamente y detectar alteraciones con precisión.

        \item \textbf{Biopsias:} Toma de muestras de tejido para análisis histopatológico.

    }

    \subsection{Métodos de Protocolos}

    Para garantizar la calidad y la uniformidad de la atención, se empleó la aplicación sistemática de protocolos.

    \subsubsection{Protocolos de Asepsia}

    En todo procedimiento quirúrgico es fundamental la aplicación de técnicas de asepsia. Esto incluye la definición de zonas de alto riesgo, el estricto respeto a la higiene de manos, el uso de equipo de protección individual y la esterilización de instrumental para prevenir infecciones.

    \subsubsection{Protocolos Anestésicos}

    Se aplicaron protocolos de anestesia que constan de tres fases: inducción, mantenimiento y reanimación. Esto incluye el cálculo seguro de dosis y la aplicación de técnicas anestésicas locorregionales (analgesia multimodal).

    \subsubsection{Protocolos de Cirugía}

    Se siguió la previsión y la preparación del material e instrumental necesario para cada intervención. Todo el instrumental quirúrgico requerido fue desinfectado y esterilizado previamente, utilizando un autoclave o esterilizadora de calor seco.

    \subsection{Métodos de Análisis y Documentación de Resultados}

    \subsubsection{Toma y Registro de Datos}

    La información obtenida producto de la aplicación de las técnicas de exploración clínica (inspección, palpación, percusión, auscultación) y la información relativa a fármacos, dosis, vías de administración y duración del tratamiento, fue centralizada y registrada en la ficha clínica o expediente clínico del paciente.

    \subsubsection{Análisis y Síntesis}

    Se empleó el \textbf{método de análisis} para descomponer la información general obtenida de la casuística de acuerdo a los objetivos específicos. El \textbf{método de síntesis} se utilizó para integrar las partes esenciales del análisis, fijar las cualidades relevantes del estudio y agrupar la información obtenida para descubrir las recomendaciones finales del trabajo.

    \subsubsection{Método Estadístico}

    Este método consistió en la secuencia de procedimientos para el manejo de los datos cualitativos y cuantitativos obtenidos de la casuística del internado. Se utilizó para la tabulación de datos y la presentación de informes, facilitando la construcción de cuadros estadísticos para exponer los resultados obtenidos en relación con el número de vacunaciones, desparasitaciones, cirugías y tratamientos realizados durante el periodo del internado.

\clearpage