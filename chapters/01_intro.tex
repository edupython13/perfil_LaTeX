\clearpage

%\newgeometry{top=2.5cm}
%\restoregeometry
%\newgeometry{left=3.5cm,right=3cm,top=2cm,bottom=3cm}
%\setcounter{page}{1}
\renewcommand{\thepage}{\arabic{page}}
\pagenumbering{arabic}
\setcounter{page}{1}
%\renewcommand{\baselinestretch}{0.95}
\section{ANTECEDENTES}
	%\addcontentsline{toc}{section}{I. INTRODUCCIÓN}
	%\textbf{ENGORDE DE CERDOS A BASE DE MAÍZ Y SORGO, CABAÑA MESA VERDE.}
	El hospital para Animales dependiente de la facultad Integral Defensores del Chaco del Municipio de Villa Vaca Guzmán, brinda atención a los propietarios de animales en prevención, diagnóstico y tratamiento de enfermedades y cirugías.\\
	Las enfermedades zoonóticas constituyen un riesgo para los seres humanos, sobre todo las enfermedades que transmiten los animales de compañía al encontrarse en contacto directo con el hombre.
	Los programas de epidemiología animal, actualmente son necesarios en todo tipo de explotación pecuaria, más aun si estas explotaciones están próximas a los centros urbanos.
	Sin embargo la salud de los animales, tanto mayores como menores casi son igual de importantes como la del ser humano, en este sentido gran parte de la población ya ha tomado conciencia de este problema, pero también hay que aceptar que existen personas que no tienen interés por sus mascotas, por mantener un animal sano y protegido de enfermedades o problemas y peligros constantes que suelen sufrir en el transcurso de su vida.\\
	Por los antecedentes anotados, en la población de Villa Vaca Guzmán municipio de la provincia Luis Calvo del departamento de Chuquisaca, se ha creado el hospital para animales que pertenece a la Facultad Integral Defensores del Chaco, con el objetivo de prestar atención médica a pacientes de diferentes especies animales, coadyuvando en el tratamiento y control de enfermedades infecciosas, parasitarias, infecto-contagiosas, anémicas y otras que se dan en la región. El hospital para animales, presta atención clínica de animales mayores y menores, cirugías, laboratorios y farmacia veterinaria, también se realizan asistencias técnicas, cursos de capacitación, seminarios en las comunidades del municipio. De esta manera, el hospital para animales también participa en la formación académica de los estudiantes, en el que los docentes utilizan los ambientes y equipamientos para realizar prácticas de las distintas materias que se dictan.

\clearpage