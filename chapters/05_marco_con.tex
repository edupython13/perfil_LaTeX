\clearpage

\section{MARCO CONTEXTUAL}
	% \subsection{Delimitación Espacial}
	% 	El presente trabajo de investigación se desarrollará en en área urbana del municipio de Villa Vaca Guzmán.
	% \subsection{Delimitación Temporal}
	% 	Gestión 2022 con una duración de 3 meses a partir del mes de septiembre.
	El presente trabajo de internado, se realizó en la población de Villa Vaca Guzmán área urbana, Villa Vaca Guzmán se encuentra ubicado en las coordenadas; \(19^{\circ}54^\prime30^{\prime\prime}\) latitud sur y \(63^{\circ}43^\prime15^{\prime\prime}\) longitud oeste, a 1117 m.s.n.m.%\vspace{-3mm}
	\subsection{Aspectos Físico Geográfico}
		%\subsubsection{Aspectos Físicos}}
		%	Hospital para Animales dependiente de la Facultad Integral Defensores del Chaco.
		\subsubsection{Ubicación Geográfica}
			El Hospital para Animales, se encuentra en barrio San José del área urbana de la población de Villa Vaca Guzmán, sobre la diagonal Jaime Mendoza a 367 km de la ciudad de Sucre entre Monteagudo y Camiri, a distancias de 53 km y 105 km respectivamente.\par
			\paragraph{Macro Localización:} Área urbana del municipio de Villa Vaca Guzmán, provincia Luis Calvo del departamento de Chuquisaca.
			\paragraph{Micro Localización:} Barrio San José, avenida Banzer, entre Facultad Integral Defensores del Chaco y unidad educativa Gerardo Vaca Guzmán.
	\subsection{Clima}
		Subtropical semi árido, con estaciones lluviosas, invierno húmedo y frío, en verano cálido con una temperatura media de 19\(^\circ\text{C}\), mínima de 3\(^\circ\text{C}\) y máxima de 42\(^\circ\text{C}\). La temperatura media anual es de 21\(^\circ\text{C}\).
	\subsection{Precipitación Fluvial}
		La precipitación media anual es de $882.2 m.m$, siendo los meses de noviembre y marzo los de mayor intensidad, presentando una humedad relativa entre 53-84\!\%.
	\subsection{Características Socio Demográficas}
		El municipio de Villa Vaca Guzmán, tiene una población de 12484 habitantes de acuerdo con los datos publicados por el Instituto Nacional de Estadística (INE) del año 2011.
		La localidad cuenta con servicios públicos; agua, alcantarillado, energía eléctrica, internet, televisión por cable, radios y telefonía celular TIGO, ENTEL y VIVA.

\clearpage