%\thispagestyle{empty}

\tocleftparagraph{RESUMEN}

\begin{center}
	\textbf{RESUMEN}
\end{center}

El presente trabajo de internado, realizado por el Univ. Luis Eduardo Miranda Barja, se llevó a cabo durante seis meses en el Hospital para Animales, dependiente de la Carrera de Medicina Veterinaria y Zootecnia de la Facultad Integral Defensores del Chaco en Villa Vaca Guzmán. El objetivo principal fue consolidar y fortalecer los conocimientos teórico-prácticos adquiridos en la formación académica, ejercitando la práctica laboral en la atención de diferentes casos clínicos y cirugías en animales menores y mayores.

Durante el desarrollo del internado, se logró la adquisición de experiencia y destrezas en diversas áreas de la clínica veterinaria, destacando los siguientes resultados:

\textbf{Diagnóstico y Tratamiento Clínico}: Se atendieron y trataron 32 casos de enfermedades virales, donde el Parvovirus Canino constituyó la patología principal, confirmando su alta prevalencia en la zona. Además, se manejaron 13 casos de intoxicaciones alimenticias y 30 casos de traumatismos y politraumatismos.

\textbf{Actividad Quirúrgica}: Se desarrollaron destrezas en la práctica quirúrgica y postoperatoria mediante la participación en 33 intervenciones quirúrgicas (incluyendo Ovariohisterectomías, Orquiectomías, y cirugías especializadas).

\textbf{Medicina Preventiva}: Se aplicaron 139 inmunizaciones (incluyendo vacunas contra Parvovirus y Heptavalente), y se realizaron 133 desparasitaciones (internas y externas). Asimismo, se trataron 18 casos de enfermedades parasitarias dermatológicas.

Se concluye que el internado fue fundamental para profundizar los conocimientos en el manejo de enfermedades infecciosas, parasitarias y quirúrgicas, logrando la experiencia necesaria y un criterio profesional propio indispensable para ejercer la profesión de médico veterinario y zootecnista.

\clearpage