% ===================================================================
% 1. PDF/A-1b: CONFIGURACIÓN INICIAL
% ===================================================================
\pdfminorversion=7
\pdfcompresslevel=9
\pdfobjcompresslevel=2
\pdfinclusioncopyfonts=1

% ===================================================================
% 2. CODIFICACIÓN, IDIOMA Y FUENTES (Times New Roman 12pt)
% ===================================================================
\usepackage[utf8]{inputenc}
\usepackage[T1]{fontenc}
\usepackage[spanish, es-noindentfirst, es-nodecimaldot]{babel}
\usepackage{newtxtext, newtxmath}
%\usepackage[newtx]
%\usepackage{microtype}
%\DisableLigatures{encoding = *, family = *}
\usepackage{textcase} % Requerido para \MakeUppercase en TOC

% ===================================================================
% 3. GEOMETRÍA DE PÁGINA (MÁRGENES ESTRICTOS)
% ===================================================================
\usepackage[
	letterpaper, left=3.5cm, right=3.0cm, top=3.0cm, bottom=3.0cm,
	%noheadfoot, headheight=16pt, headsep=10pt,
	heightrounded, showframe=false
]{geometry}

% Posiciona la numeración a 1.5 cm del borde inferior
\setlength{\footskip}{1.5cm}

% ===================================================================
% 4. ESPACIADO Y PÁRRAFOS (Interlineado 1.5)
% ===================================================================
\usepackage{setspace}
\onehalfspacing
%\renewcommand{\arraystretch}{1.5}
%\doublespacing
\usepackage{parskip}
%\setlength{\parindent}{0pt}
\pagestyle{plain}
\raggedbottom
%\sloppy

% ===================================================================
% 5. PAQUETES DE TABLAS Y ESTRUCTURA
% ===================================================================
\usepackage{graphicx}
\usepackage{float}
\usepackage{enumitem}
\usepackage{multirow}
%\usepackage{rotating}
\usepackage{longtable}
\usepackage{array}
%\usepackage{tabularx}
%\usepackage{xltabular}
\usepackage{booktabs}
\usepackage{pdflscape}
\usepackage[table]{xcolor}
\usepackage{ragged2e}
\usepackage{caption}
\captionsetup{
	font=normalsize,
	labelfont=bf,
	%justification=centering,
	labelsep=period,
	skip=6pt
}
\usepackage{multicol}
\usepackage{calc}

\usepackage{ifthen} % Mantenido para funcionalidad de \pdft

\newcolumntype{J}{>{\Justifying\arraybackslash}X}
\newcolumntype{C}{>{\Centering\arraybackslash}X}


\usepackage{bookmark}

% ===================================================================
% 6. TÍTULOS, TOC Y BIBLIOGRAFÍA
% ===================================================================
\usepackage[compact]{titlesec}
\usepackage[nottoc, numbib]{tocbibind}
\usepackage[fixlanguage]{babelbib}
\selectbiblanguage{spanish}
\usepackage[numbers, sort&compress]{natbib}
\usepackage{tocloft}

% === ÍNDICE GENERAL (CORRECTO, CENTRADO, MAYÚSCULAS, ESTILO GLOBAL) ===
\addto\captionsspanish{%
	\renewcommand{\contentsname}{\centering\normalsize\bfseries ÍNDICE GENERAL}%
}
\renewcommand{\cfttoctitlefont}{\centering\normalsize\bfseries}
%\renewcommand{\cftaftertoctitle}{\\[1.5\baselineskip]}% Respeta interlineado 1.5

% Formato de títulos de sección normales
\titleformat{\section}[hang]{\normalsize\bfseries}{\thesection}{1em}{}
\titleformat{\subsection}{\normalsize\bfseries}{\thesubsection}{1em}{}
\titleformat{\subsubsection}{\normalsize\bfseries}{\thesubsubsection}{1em}{}

% Numeración con punto final
\setcounter{secnumdepth}{6}
\setcounter{tocdepth}{4}
\renewcommand{\thesection}{\Roman{section}.}
\renewcommand{\thesubsection}{\arabic{section}.\arabic{subsection}.}
\renewcommand{\thesubsubsection}{\arabic{section}.\arabic{subsection}.\arabic{subsubsection}.}
\renewcommand{\theparagraph}{\arabic{section}.\arabic{subsection}.\arabic{subsubsection}.\arabic{paragraph}}
\renewcommand{\thesubparagraph}{\arabic{section}.\arabic{subsection}.\arabic{subsubsection}.\arabic{paragraph}.\arabic{subparagraph}}

% Indentación del TOC
\cftsetindents{section}{0em}{2.3em}
\cftsetindents{subsection}{2.3em}{2.3em}
\cftsetindents{subsubsection}{4.6em}{3.3em}
\cftsetindents{paragraph}{7.9em}{4.6em}
\cftsetindents{subparagraph}{12.5em}{4.8em}

% ===================================================================
% 7. COMANDOS PERSONALIZADOS (Optimizados)
% ===================================================================
\newcommand{\pdf}[3]{%
	\begin{center}%
		\includegraphics[scale=#1, page=#2]{#3}%
	\end{center}%
}

\newcommand{\pdft}[5]{%
	\begin{center}%
		\ifthenelse{\equal{#5}{true}}{%
			\includegraphics[scale=#1, page=#2, trim=#4, clip]{#3}%
		}{%
			\includegraphics[scale=#1, page=#2]{#3}%
		}%
	\end{center}%
}

\newcommand{\xcite}[1]{(\citeauthor{#1}, \citeyear{#1})}

\newcounter{tocleftparcounter}
\makeatletter
\newcommand{\tocleftparagraph}[1]{%
	\stepcounter{tocleftparcounter}%
	\edef\tocleftparlabel{tocleftpar.\thetocleftparcounter}%
	\hypertarget{\tocleftparlabel}{}%
	
	\addtocontents{toc}{%
		\protect\normalsize\protect\l@section{%
			\protect\hyperlink{\tocleftparlabel}{%
				\protect\RaggedRight\protect\textbf{#1}\protect\cftsecleader%
			}%
		}{%
			\protect\hyperlink{\tocleftparlabel}{%
				\protect\cftsecpagefont\protect\textbf{\thepage}%
			}%
		}%
	}%
}
\makeatother

\newcommand{\leftitemize}[1]{%
	\begin{itemize}[leftmargin=*, labelsep=4pt, itemindent=10pt, topsep=0pt, partopsep=0pt, parsep=0pt]%
		#1%
	\end{itemize}%
}

\newcommand{\leftenumerate}[1]{%
	\begin{enumerate}[leftmargin=*, labelsep=4pt, itemindent=12pt, topsep=0pt, partopsep=0pt, parsep=0pt, label=\arabic{enumi}.]%
		#1%
	\end{enumerate}%
}

\providecommand{\tightlist}{\setlength{\itemsep}{0pt}\setlength{\parskip}{0pt}}

% ===================================================================
% 8. HYPERREF (SIEMPRE AL FINAL)
% ===================================================================
\usepackage{hyperref}
\hypersetup{
	pdftitle={Clínica de Animales Menores y Mayores},
	pdfauthor={Luis Eduardo Miranda Barja},
	pdfsubject={Medicina Veterinaria y Zootecnia},
	pdfkeywords={Veterinaria, Zootecnia, Internado, Clínica},
	pdfcreator={LaTeX con pdfTeX},
	pdfproducer={TeX Live 2025},
	hidelinks,
	colorlinks=false,
	linkcolor=black,
	citecolor=black,
	urlcolor=gray,
	%bookmarks=true,
	bookmarksopen=true,
	bookmarksnumbered=true,
	pdfpagemode=UseOutlines,
	pdfstartview=FitH,
	pdfborder={0 0 0}
}

% -------------------------------
% 9. PDF/A-1b FINAL
% -------------------------------
\pdfinfo{
	/Title (Clínica de Animales Menores y Mayores)
	/Author (Luis Eduardo Miranda Barja)
	/Subject (Medicina Veterinaria y Zootecnia)
	/Keywords (Veterinaria, Zootecnia, Internado, Clínica)
}

% -------------------------------
% 10. ROMANOS EN MINÚSCULAS (CORREGIDO SIN ERRORES)
% -------------------------------
\makeatletter
% 1. Redefinición de la primitiva interna para asegurar minúsculas
% La primitiva \romannumeral produce minúsculas.
\def\@roman#1{\romannumeral #1}

% 2. Opcional: Si usa hyperref (recomendado para PDFs), 
% parchee la referencia para el TOC, asegurando minúsculas 
% sin depender de \thepage que podría estar redefinida.
% Nota: El paquete etoolbox debe cargarse primero si usa \patchcmd.
\usepackage{etoolbox}
\patchcmd{\Hy@writebookmark}{\thepage}{\roman{page}}{}{}
\makeatother