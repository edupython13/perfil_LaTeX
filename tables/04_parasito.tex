% 04_parasitaria.tex
\begin{landscape}

%\section*{REGISTRO DE ENFERMEDADES PARASITARIAS}
%\vspace{0.3cm}

% =============================================================================
% CONFIGURACIÓN (basada en tu modelo)
% =============================================================================
\setlength{\tabcolsep}{1pt}
\fontsize{10pt}{10pt}\selectfont
\setlength{\extrarowheight}{3pt}

% =============================================================================
% LONGTABLE CON CAPTION AL PRINCIPIO
% =============================================================================
\begin{longtable}{
  |>{\centering\arraybackslash}m{0.08\linewidth}           % ESPECIE
  |>{\centering\arraybackslash}m{0.08\linewidth}           % Nº CASOS
  |>{\centering\arraybackslash}m{0.14\linewidth}           % ENFERMEDADES
  |>{\raggedright\arraybackslash}m{0.10\linewidth}          % FÁRMACO
  |>{\raggedright\arraybackslash}m{0.24\linewidth}          % PRINCIPIO ACTIVO
  |>{\raggedright\arraybackslash}m{0.14\linewidth}          % DOSIS/VÍA
  |>{\centering\arraybackslash}m{0.06\linewidth}            % TIEMPO
  |>{\raggedright\arraybackslash}m{0.14\linewidth}|         % OBSERVACIONES
}

% ──────────────── CAPTION COMO TÍTULO (solo en primera página) ────────────────
\caption{Registro de Tratamientos para Enfermedades Parasitarias}
\label{tab:parasitarias}
\endfirsthead

\caption[]{(Continuación) Registro de Tratamientos para Enfermedades Parasitarias}

% ──────────────── ENCABEZADO PARA PÁGINAS SIGUIENTES ────────────────
\hline
\rowcolor{blue!10}
\textbf{ESPECIE} & \textbf{Nº DE CASOS} & \textbf{ENFERMEDADES} & \textbf{FÁRMACO} & 
\textbf{PRINCIPIO ACTIVO} & \textbf{DOSIS/VÍA DE ADMINISTRACIÓN} & 
\textbf{DURA\-CIÓN} & \textbf{OBSERVACIONES} \\
\hline
\endhead

% ──────────────── PIE DE PÁGINA (continuación) ────────────────
\hline
\multicolumn{8}{r}{\footnotesize\textit{Continúa en la siguiente página \ldots}}
\endfoot

% ──────────────── PIE FINAL ────────────────
\hline
\multicolumn{8}{l}{\footnotesize\textit{\textbf{Fuente}: Elaboración propia.}}
\endlastfoot

% ──────────────── ENCABEZADO PRIMERA PÁGINA (con color) ────────────────
\hline
\rowcolor{blue!10}
\textbf{ESPECIE} & \textbf{Nº DE CASOS} & \textbf{ENFERMEDADES} & \textbf{FÁRMACO} & 
\textbf{PRINCIPIO ACTIVO} & \textbf{DOSIS/VÍA DE ADMINISTRACIÓN} & 
\textbf{DURA\-CIÓN} & \textbf{OBSERVACIONES} \\
\hline

% =============================================================================
% CONTENIDO DE LA TABLA (extraído del PDF '04_parasitaria.pdf')
% =============================================================================

% Canino – Sarna sarcoptica
\multirow{6}{*}{\textbf{Canino}} & \multirow{6}{*}{17} & \multirow{6}{*}{Sarna sarcoptica} &
Ranger & Ivermectina 1\% & 1ml/30kg SC & \multirow{6}{*}{3 días} & \multirow{6}{*}{Satisfactorio} \\
\cline{4-6}
& & & Galerfin & Clorfeniramina maleto & 0,2-2ml IV-IM & & \\
\cline{4-6}
& & & Atriben & Triamcinolona & 1ml/20kg IM-SC & & \\
\cline{4-6}
& & & Cefavet & Cefalexina & 1ml/20kg IM-SC & & \\
\cline{4-6}
& & & Dermo vet & Benzoato de bencilo, boro salicílico & Tópico & & \\
\cline{4-6}
& & & Impacto & Cipermetrina, clorpirifos, citronelal & 1ml/1litro de agua & & \\
\hline

% Canino – Sarna demodecica
\multirow{5}{*}{\textbf{Canino}} & \multirow{5}{*}{1} & \multirow{5}{*}{Sarna demodecica} &
Ranger & Ivermectina 1\% & 1ml/30kg SC & \multirow{5}{*}{3 días} & \multirow{5}{*}{Satisfactorio} \\
\cline{4-6}
& & & Atriben & Triamcinolona & 1ml/20kg IM-SC & & \\
\cline{4-6}
& & & Cefavet & Cefalexina & 1ml/20kg IM-SC & & \\
\cline{4-6}
& & & Dermo vet & Benzoato de bencilo, boro salicílico & Tópico & & \\
\cline{4-6}
& & & Impacto & Cipermetrina, clorpirifos, citronelal & 1ml/1litro de agua & & \\
\hline

\end{longtable}
\end{landscape}