\begin{landscape}

%\subsection*{Cuadro Nº 4: Enfermedades parasitarias}
%\vspace{0.3cm}

% =============================================================================
% CONFIGURACIÓN TIPOGRÁFICA (NO TOCAR – FUNCIONA PERFECTAMENTE)
% =============================================================================
%\fontsize{10pt}{15pt}\selectfont
%\setlength{\extrarowheight}{2pt}
%\setlength{\tabcolsep}{3pt}
\setlength{\tabcolsep}{3pt}
\fontsize{10pt}{10pt}\selectfont
\setlength{\extrarowheight}{2pt}

% =============================================================================
% TABLA
% =============================================================================
\begin{longtable}{
  |>{\centering\arraybackslash}m{0.10\linewidth}
  |>{\centering\arraybackslash}m{0.05\linewidth}
  |>{\centering\arraybackslash}m{0.13\linewidth}
  |>{\raggedright\arraybackslash}m{0.13\linewidth}
  |>{\raggedright\arraybackslash}m{0.18\linewidth}
  |>{\raggedright\arraybackslash}m{0.14\linewidth}
  |>{\centering\arraybackslash}m{0.06\linewidth}
  |>{\raggedright\arraybackslash}m{0.16\linewidth}|
}

% -----------------------------------------------------------------------------
% CAPTION Y LABEL
% -----------------------------------------------------------------------------
\caption[]{Tratamiento de enfermedades parasitarias}
\label{tab:enfermedades_parasitarias}\\
\endfirsthead

% -----------------------------------------------------------------------------
% ENCABEZADO (PRIMERA PÁGINA Y SIGUIENTES)
% -----------------------------------------------------------------------------
\hline
\rowcolor{blue!10}
\textbf{ESPECIE} &
\textbf{Nº CASOS} &
\textbf{ENFERMEDAD} &
\textbf{FÁRMACO} &
\textbf{PRINCIPIO ACTIVO} &
\textbf{DOSIS/VÍA de ADMINISTRACIÓN} &
\textbf{TIEMPO} &
\textbf{OBSERVACIONES} \\
\hline
\endhead

% -----------------------------------------------------------------------------
% PIE DE CONTINUACIÓN
% -----------------------------------------------------------------------------
\multicolumn{8}{r}{\footnotesize\textit{Continúa en la siguiente página}} \\
\endfoot
\hline

\endlastfoot

\hline
\rowcolor{blue!10}
\textbf{ESPECIE} &
\textbf{Nº CASOS} &
\textbf{ENFERMEDAD} &
\textbf{FÁRMACO} &
\textbf{PRINCIPIO ACTIVO} &
\textbf{DOSIS/VÍA de ADMINISTRACIÓN} &
\textbf{TIEMPO} &
\textbf{OBSERVACIONES} \\
\hline

% =============================================================================
% CANINO – SARNA SARCÓPTICA
% =============================================================================
\multirow{6}{*}{\textbf{Canino}} &
\multirow{6}{*}{17} &
\multirow{6}{*}{Sarna sarcóptica} &
Ranger & Ivermectina 1\% & 1\,ml/30\,kg SC &
\multirow{6}{*}{3 días} &
\multirow{6}{*}{Satisfactorio} \\
\cline{4-6}
& & & Galerfin & Clorfeniramina maleato & 0,2-2\,ml IV-IM & & \\
\cline{4-6}
& & & Atriben & Triamcinolona & 1\,ml/20\,kg IM-SC & & \\
\cline{4-6}
& & & Cefavet & Cefalexina & 1\,ml/20\,kg IM-SC & & \\
\cline{4-6}
& & & Dermo vet & Benzoato de bencilo, boro salicílico & Tópico & & \\
\cline{4-6}
& & & Impacto & Cipermetrina, clorpirifos, citronelal & 1\,ml/1\,litro de agua & & \\
\hline

% =============================================================================
% CANINO – SARNA DEMODÉCICA
% =============================================================================
\multirow{5}{*}{\textbf{Canino}} &
\multirow{5}{*}{1} &
\multirow{5}{*}{Sarna demodécica} &
Ranger & Ivermectina 1\% & 1\,ml/30\,kg SC &
\multirow{5}{*}{3 días} &
\multirow{5}{*}{Satisfactorio} \\
\cline{4-6}
& & & Atriben & Triamcinolona & 1\,ml/20\,kg IM-SC & & \\
\cline{4-6}
& & & Cefavet & Cefalexina & 1\,ml/20\,kg IM-SC & & \\
\cline{4-6}
& & & Dermo vet & Benzoato de bencilo, boro salicílico & Tópico & & \\
\cline{4-6}
& & & Impacto & Cipermetrina, clorpirifos, citronelal & 1\,ml/1\,litro de agua & & \\
\hline

\end{longtable}
\end{landscape}