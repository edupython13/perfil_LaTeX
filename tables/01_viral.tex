\begin{landscape}

\section{RESULTADOS}

Los resultados se ven reflejados en los cuadros durante los 6 meses de internado en el Hospital para Animales de la Facultad Integral Defensores del Chaco.

%\subsection*{Cuadro Nº 1: Enfermedades virales}
%\vspace{0.3cm}


% =============================================================================
% CONFIGURACIÓN (exactamente como tú la tenías funcionando)
% =============================================================================
% \newgeometry{...}        % lo dejas comentado si prefieres que lo controle el documento principal
% \renewcommand{\arraystretch}{1.5}   % comentado en tu versión original
\setlength{\tabcolsep}{3pt}
\fontsize{10pt}{10pt}\selectfont
%\setlength{\tabcolsep}{3pt}
\setlength{\extrarowheight}{2pt}

% =============================================================================
% LONGTABLE CON CAPTION AL PRINCIPIO (como título real de tabla)
% =============================================================================
\begin{longtable}{
  |>{\centering\arraybackslash}p{0.10\linewidth}
  |>{\centering\arraybackslash}p{0.05\linewidth}
  |>{\centering\arraybackslash}p{0.13\linewidth}
  |>{\raggedright\arraybackslash}p{0.13\linewidth}
  |>{\raggedright\arraybackslash}p{0.18\linewidth}
  |>{\raggedright\arraybackslash}p{0.14\linewidth}
  |>{\centering\arraybackslash}p{0.06\linewidth}
  |>{\raggedright\arraybackslash}p{0.16\linewidth}|
}

% ──────────────── CAPTION COMO TÍTULO (solo en primera página) ────────────────
\caption{Tratamiento de enfermedades infecciosas virales}%
\label{tab:enfermedades_infecciosas}\\
\endfirsthead

% ──────────────── ENCABEZADO PARA PÁGINAS SIGUIENTES ────────────────
\hline
\rowcolor{blue!10}
\textbf{ESPECIE} & \textbf{Nº CASOS} & \textbf{ENFERMEDAD} & \textbf{FÁRMACO} &
\textbf{PRINCIPIO ACTIVO} & \textbf{DOSIS/VÍA de ADMINISTRACIÓN} &
\textbf{TIEMPO} & \textbf{OBSERVACIONES} \\
\hline
\endhead

% ──────────────── PIE DE PÁGINA (continuación) ────────────────
\hline
\multicolumn{8}{r}{\footnotesize\textit{Continúa en la siguiente página}} \\
\endfoot

% ──────────────── PIE FINAL (nada aquí para que no se repita el caption) ────────────────
\endlastfoot

% ──────────────── ENCABEZADO PRIMERA PÁGINA (con color) ────────────────
\hline
\rowcolor{blue!10}
\textbf{ESPECIE} & \textbf{Nº CASOS} & \textbf{ENFERMEDAD} & \textbf{FÁRMACO} &
\textbf{PRINCIPIO ACTIVO} & \textbf{DOSIS/VÍA de ADMINISTRACIÓN} &
\textbf{TIEMPO} & \textbf{OBSERVACIONES} \\
\hline

% =============================================================================
% CONTENIDO DE LA TABLA (sin ningún cambio respecto a tu versión original)
% =============================================================================

% CANINO – PARVOVIRUS
\multirow{7}{*}{\textbf{Canino}} & \multirow{7}{*}{14} & \multirow{7}{*}{Parvovirus} &
Ringer Lactato & Electrolitos & Según criterio médico IV & \multirow{7}{*}{3-5 días} &
\multirow{7}{*}{\parbox{\linewidth}{2 pacientes no respondieron al tratamiento}} \\
\cline{4-6}
& & & Ditral & Sulfadoxina, trimetoprim & 1\,ml/10\,kg IV-IM & & \\
\cline{4-6}
& & & Clortetrason & Oxitetraciclina clorhidrato, florfenicol, prednisolona, lidocaína & 1\,ml/5\,kg IM & & \\
\cline{4-6}
& & & Desalgina & Dipirona & 1\,ml/10\,kg IV-IM-SC & & \\
\cline{4-6}
& & & Complejo B & Vitamina B1, B2, B6, B12, B15 & 1\,ml/20\,kg IM-SC & & \\
\cline{4-6}
& & & AminoLab Forte & Aminoácidos, dextrosa, electrolitos, vitaminas & 1\,ml/1\,kg IV & & \\
\cline{4-6}
& & & Ranitidina & Ranitidina & 1\,ml/5\,kg IV-IM & & \\
\hline

% CANINO – DISTEMPER
\multirow{5}{*}{\textbf{Canino}} & \multirow{5}{*}{5} & \multirow{5}{*}{Distemper} &
Ringer Lactato & Electrolitos & Según criterio médico IV & \multirow{5}{*}{3-5 días} &
\multirow{5}{*}{\parbox{\linewidth}{1 paciente no respondió al tratamiento}} \\
\cline{4-6}
& & & Desalgina & Dipirona & 1\,ml/10\,kg IV-IM-SC & & \\
\cline{4-6}
& & & AminoLab Forte & Aminoácidos, dextrosa, electrolitos, vitaminas & 1\,ml/1\,kg IV & & \\
\cline{4-6}
& & & Hepato-Ject & Ácido ticético, aminoácidos, vitaminas & 1\,ml/20\,kg IV-IM & & \\
\cline{4-6}
& & & Floxa Benectimida & Enrofloxacina y clorhidrato de benectimida & 1\,ml/10\,kg IM & & \\
\hline

% CANINO – HEPATITIS
\multirow{3}{*}{\textbf{Canino}} & \multirow{3}{*}{7} & \multirow{3}{*}{Hepatitis} &
Complejo B & Vitamina B1, B2, B6, B12, B15 & 1\,ml/20\,kg IM-SC & \multirow{3}{*}{3-5 días} &
\multirow{3}{*}{Satisfactorio} \\
\cline{4-6}
& & & Clortetrason & Oxitetraciclina, florfenicol, prednisolona, lidocaína & 1\,ml/5\,kg IM & & \\
\cline{4-6}
& & & Mercepton & Aminoácidos, vitaminas, colina & 1\,ml/10\,kg IV-IM-SC & & \\
\hline

% FELINO – PANLEUCOPENIA
\multirow{4}{*}{\textbf{Felino}} & \multirow{4}{*}{1} & \multirow{4}{*}{Panleucopenia} &
Ringer Lactato & Electrolitos & Según criterio médico IV & \multirow{4}{*}{3-5 días} &
\multirow{4}{*}{Satisfactorio} \\
\cline{4-6}
& & & Complejo B & Vitamina B1, B2, B6, B12, B15 & 1\,ml/20\,kg IM-SC & & \\
\cline{4-6}
& & & Clortetrason & Oxitetraciclina, florfenicol, prednisolona, lidocaína & 1\,ml/5\,kg IM & & \\
\cline{4-6}
& & & Ditral & Sulfadoxina, trimetoprim & 1\,ml/10\,kg IV-IM & & \\
\hline

\end{longtable}

% =============================================================================
% ELIMINASTE EL \begin{table}...\end{table} QUE TENÍAS ANTES
% =============================================================================
% \restoregeometry   % descomenta si usaste \newgeometry antes

\end{landscape}