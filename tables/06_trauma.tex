% 06_tramatismo.tex
\begin{landscape}

	%\section*{REGISTRO DE TRAUMATISMOS}
	%\vspace{0.3cm}

	% =============================================================================
	% CONFIGURACIÓN (basada en tu modelo)
	% =============================================================================
	\setlength{\tabcolsep}{1pt}
	\setlength{\extrarowheight}{3pt}
	\fontsize{10pt}{10pt}\selectfont

	% =============================================================================
	% CÁLCULO DE ANCHOS DE COLUMNAS
	% =============================================================================
	% Total de ancho disponible: \linewidth (en landscape es el ancho horizontal)
	% Restamos el espacio ocupado por líneas verticales y espacios entre columnas
	% 8 columnas = 7 líneas verticales + espacios entre columnas
	\newlength{\availablewidth}
	\setlength{\availablewidth}{\linewidth}
	\addtolength{\availablewidth}{-14\tabcolsep} % 7 espacios internos * 2
	\addtolength{\availablewidth}{-0.4pt} % Ancho de líneas verticales

	% =============================================================================
	% LONGTABLE CON CAPTION AL PRINCIPIO
	% =============================================================================
	\begin{longtable}{
		|>{\centering\arraybackslash}m{0.08\availablewidth}           % ESPECIE
		|>{\centering\arraybackslash}m{0.07\availablewidth}           % Nº CASOS
		|>{\centering\arraybackslash}m{0.14\availablewidth}           % ENFERMEDADES
		|>{\raggedright\arraybackslash}m{0.12\availablewidth}          % FÁRMACO
		|>{\raggedright\arraybackslash}m{0.24\availablewidth}          % PRINCIPIO ACTIVO
		|>{\raggedright\arraybackslash}m{0.152\availablewidth}          % DOSIS/VÍA
		|>{\centering\arraybackslash}m{0.07\availablewidth}            % TIEMPO
		|>{\raggedright\arraybackslash}m{0.12\availablewidth}|         % OBSERVACIONES
	}

		% ──────────────── CAPTION COMO TÍTULO (solo en primera página) ────────────────
		\caption{\raggedright Registro de Tratamientos para Traumatismos}
		\label{tab:traumatismos} \\
		\hline
		\rowcolor{blue!10}
		\textbf{ESPECIE} & \textbf{Nº DE CASOS} & \textbf{ENFERMEDADES} & \textbf{FÁRMACO} & 
		\textbf{PRINCIPIO ACTIVO} & \textbf{DOSIS/VÍA DE ADMINISTRACIÓN} & 
		\textbf{DURA\-CIÓN} & \textbf{OBSERVACIO\-NES} \\
		\hline
		\endfirsthead

		% ──────────────── ENCABEZADO PARA PÁGINAS SIGUIENTES ────────────────
		\multicolumn{8}{l}{\normalsize\textbf{(Continuación) Cuadro \ref{tab:traumatismos}}. Registro de Tratamientos para Traumatismos} \\[6pt]
		\hline
		\rowcolor{blue!10}
		\textbf{ESPECIE} & \textbf{Nº DE CASOS} & \textbf{ENFERMEDADES} & \textbf{FÁRMACO} & 
		\textbf{PRINCIPIO ACTIVO} & \textbf{DOSIS/VÍA DE ADMINISTRACIÓN} & 
		\textbf{DURA\-CIÓN} & \textbf{OBSERVACIO\-NES} \\
		\hline
		\endhead

		% ──────────────── PIE DE PÁGINA (continuación) ────────────────
		\hline
		\multicolumn{8}{r}{\footnotesize\textit{Continúa en la siguiente página \ldots}}
		\endfoot

		% ──────────────── PIE FINAL ────────────────
		\hline
		\multicolumn{8}{l}{\footnotesize\textit{\textbf{Fuente}: Elaboración propia.}}
		\endlastfoot

		% =============================================================================
		% CONTENIDO DE LA TABLA (extraído del PDF '06_tramatismo.pdf')
		% =============================================================================

		% Canino – Traumatismo (fractura)
		\multirow{7}{*}{\textbf{Canino}} & \multirow{7}{*}{4} & \multirow{7}{*}{\parbox{\linewidth}{\centering{Traumatismo\\(fractura)}}} &
		Xilacina & Xilacina & 1ml/20kg IV-IM & \multirow{7}{*}{\parbox{\linewidth}{3-4 semanas}} &  \\
		\cline{4-6}
		& & & Lidocaina & Lidocaina & Según criterio médico & & \\
		\cline{4-6}
		& & & Oximed & Oxitetraciclina, bencidamina & 1ml/5kg IM-SC & & \\
		\cline{4-6}
		& & & Desalgina & Dipirona & 1ml/10kg IV-IM-SC & & \\
		\cline{4-6}
		& & & Agua oxigenada & Peróxido de hidrógeno & Tópico & & \\
		\cline{4-6}
		& & & Yodo podovidona & Complejo de yodo molecular con podovidona & Tópico & & \\
		\cline{4-6}
		& & & Galmetrim plus & Cipermetrina, imidacloprid, sulfadiazina de plata & Local externo & & \\
		\hline

		% Canino – Politraumatismo
		\multirow{4}{*}{\textbf{Canino}} & \multirow{4}{*}{24} & \multirow{4}{*}{Politraumatismo} &
		Desalgina & Dipirona & 1ml/10kg IV-IM-SC & \multirow{4}{*}{3-5 días} & \multirow{4}{*}{Satisfactorio} \\
		\cline{4-6}
		& & & Oximed & Oxitetraciclina, bencidamina & 1ml/5kg IM-SC & & \\
		\cline{4-6}
		& & & Complejo B & Vitamina B1, Vitamina B2, Vitamina B6, Vitamina B12, Vitamina B15 & 1ml/20kg IM-SC & & \\
		\cline{4-6}
		& & & Diclofenaco & Diclofenaco & 1ml/20kg IV-IM & & \\
		\hline

		% Felino – Contusión
		\multirow{4}{*}{\textbf{Felino}} & \multirow{4}{*}{2} & \multirow{4}{*}{Contusión} &
		Ankofen & Ketoprofeno & 0,5ml/25kg IM & \multirow{4}{*}{3-5 días} & \multirow{4}{*}{Satisfactorio} \\
		\cline{4-6}
		& & & Agua oxigenada & Peróxido de hidrógeno & Tópico & & \\
		\cline{4-6}
		& & & Yodo podovidona & Complejo de yodo molecular con podovidona & Tópico & & \\
		\cline{4-6}
		& & & Emplasto cicatrizante & Ácido galotanico, podovidona, yodo metálico, yodo de potasio & Tópico & & \\
		\hline

	\end{longtable}
\end{landscape}
