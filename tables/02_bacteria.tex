\begin{landscape}

\subsection*{Cuadro Nº 2: Enfermedades bacterianas}
\vspace{0.3cm}

% =============================================================================
% CONFIGURACIÓN TIPOGRÁFICA (NO TOCAR – FUNCIONA PERFECTAMENTE)
% =============================================================================
\fontsize{10pt}{15pt}\selectfont
\setlength{\extrarowheight}{2pt}
\setlength{\tabcolsep}{3pt}

% =============================================================================
% TABLA
% =============================================================================
\begin{longtable}{
  |>{\centering\arraybackslash}m{0.10\linewidth}
  |>{\centering\arraybackslash}m{0.05\linewidth}
  |>{\centering\arraybackslash}m{0.13\linewidth}
  |>{\raggedright\arraybackslash}m{0.13\linewidth}
  |>{\raggedright\arraybackslash}m{0.18\linewidth}
  |>{\raggedright\arraybackslash}m{0.14\linewidth}
  |>{\centering\arraybackslash}m{0.06\linewidth}
  |>{\raggedright\arraybackslash}m{0.16\linewidth}|
}

% -----------------------------------------------------------------------------
% ENCABEZADO (PRIMERA PÁGINA Y SIGUIENTES)
% -----------------------------------------------------------------------------
\hline
\rowcolor{blue!10}
\textbf{ESPECIE} &
\textbf{Nº CASOS} &
\textbf{ENFERMEDAD} &
\textbf{FÁRMACO} &
\textbf{PRINCIPIO ACTIVO} &
\textbf{DOSIS/VÍA de ADMINISTRACIÓN} &
\textbf{TIEMPO} &
\textbf{OBSERVACIONES} \\
\hline
\endhead

% -----------------------------------------------------------------------------
% PIE DE CONTINUACIÓN
% -----------------------------------------------------------------------------
\hline
\multicolumn{8}{r}{\footnotesize\textit{Continúa en la siguiente página}} \\
\endfoot

% -----------------------------------------------------------------------------
% CAPTION Y LABEL
% -----------------------------------------------------------------------------
\caption[]{Tratamiento de enfermedades infecciosas bacterianas}
\label{tab:enfermedades_bacterianas}
\endlastfoot

% =============================================================================
% CANINO – TRAQUEO-BRONQUITIS
% =============================================================================
\multirow{4}{*}{\textbf{Canino}} &
\multirow{4}{*}{15} &
\multirow{4}{*}{Traqueo-bronquitis} &
Complejo yodado & Yoduro de potasio, yodo metálico & 0,3\,ml/10\,kg IM &
\multirow{4}{*}{3-5 días} &
\multirow{4}{*}{Satisfactorio} \\
\cline{4-6}
& & & Bromexan & Brohexina & 1-3\,ml IM & & \\
\cline{4-6}
& & & Desalgina & Dipirona & 1\,ml/10\,kg IV-IM-SC & & \\
\cline{4-6}
& & & Interflox & Enrofloxacino & 1\,ml/10\,kg IM-SC & & \\
\hline

% =============================================================================
% CANINO – GASTROENTERITIS
% =============================================================================
\multirow{4}{*}{\textbf{Canino}} &
\multirow{4}{*}{29} &
\multirow{4}{*}{Gastroenteritis} &
Gentamicin & Gentamicina, metilparabeno, propilparabeno & 1\,ml/20\,kg IV-IM &
\multirow{4}{*}{3-5 días} &
\multirow{4}{*}{Satisfactorio} \\
\cline{4-6}
& & & AminoLab Forte & Aminoácidos, dextrosa, cloruro de potasio, cloruro de sodio, cloruro de calcio, acetato de sodio, vitamina B1 B2 B6 B12 & 1\,ml/1\,kg IV & & \\
\cline{4-6}
& & & Complejo B & Vitamina B1, Vitamina B2, Vitamina B6, Vitamina B12, Vitamina B15 & 1\,ml/20\,kg IM-SC & & \\
\cline{4-6}
& & & Hepato-Ject & Ácido tioctico, ácido orótico, DL-Metionina/N-Acetil-L-Metionina, D-pantenol, Piridoxina HCl, ácido fólico, cloruro de colina, inositol, betaina HCl & 1\,ml/20\,kg IV-IM & & \\
\hline

% =============================================================================
% CANINO – CONJUNTIVITIS
% =============================================================================
\multirow{1}{*}{\textbf{Canino}} &
\multirow{1}{*}{10} &
\multirow{1}{*}{Conjuntivitis} &
Dexaflox & Colirio, ofloxacina, dexametasona fosfato & 1-2 gotas V.Oftálmica &
\multirow{1}{*}{3-5 días} &
\multirow{1}{*}{Satisfactorio} \\
\hline

% =============================================================================
% CANINO – OTITIS
% =============================================================================
\multirow{2}{*}{\textbf{Canino}} &
\multirow{2}{*}{7} &
\multirow{2}{*}{Otitis} &
Atriben & Triamcinolona & 1\,ml/20\,kg IM-SC &
\multirow{2}{*}{3-5 días} &
\multirow{2}{*}{Satisfactorio} \\
\cline{4-6}
& & & Opter & Enrofloxacina, clotrimazol, betametasona, ivermectina & 6-12 gotas V.Ótica & & \\
\hline

% =============================================================================
% FELINO – TRAQUEO-BRONQUITIS
% =============================================================================
\multirow{3}{*}{\textbf{Felino}} &
\multirow{3}{*}{3} &
\multirow{3}{*}{Traqueo-bronquitis} &
Complejo yodado & Yoduro de potasio, yodo metálico & 0,3\,ml/10\,kg IM &
\multirow{3}{*}{3-5 días} &
\multirow{3}{*}{Satisfactorio} \\
\cline{4-6}
& & & Bromexan & Brohexina & 1-3\,ml IM & & \\
\cline{4-6}
& & & Interflox & Enrofloxacino & 1\,ml/10\,kg IM-SC & & \\
\hline

% =============================================================================
% FELINO – CONJUNTIVITIS
% =============================================================================
\multirow{1}{*}{\textbf{Felino}} &
\multirow{1}{*}{4} &
\multirow{1}{*}{Conjuntivitis} &
Dexaflox & Colirio, ofloxacina, dexametasona fosfato & 1-2 gotas V.Oftálmica &
\multirow{1}{*}{3-5 días} &
\multirow{1}{*}{Satisfactorio} \\
\hline

% =============================================================================
% FELINO – OTITIS
% =============================================================================
\multirow{2}{*}{\textbf{Felino}} &
\multirow{2}{*}{1} &
\multirow{2}{*}{Otitis} &
Atriben & Triamcinolona & 1\,ml/20\,kg IM-SC &
\multirow{2}{*}{3-5 días} &
\multirow{2}{*}{Satisfactorio} \\
\cline{4-6}
& & & Opter & Enrofloxacina, clotrimazol, betametasona, ivermectina & 6-12 gotas V.Ótica & & \\
\hline

\end{longtable}
\end{landscape}