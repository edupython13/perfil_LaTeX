% 02_bacteriana.tex
\begin{landscape}

	%\section*{REGISTRO DE ENFERMEDADES BACTERIANAS}
	%\vspace{0.3cm}

	% =============================================================================
	% CONFIGURACIÓN (basada en tu modelo)
	% =============================================================================
	\setlength{\tabcolsep}{1pt}
	\setlength{\extrarowheight}{3pt}
	\fontsize{10pt}{10pt}\selectfont

	% =============================================================================
	% CÁLCULO DE ANCHOS DE COLUMNAS
	% =============================================================================
	% Total de ancho disponible: \linewidth (en landscape es el ancho horizontal)
	% Restamos el espacio ocupado por líneas verticales y espacios entre columnas
	% 8 columnas = 7 líneas verticales + espacios entre columnas
	%\newlength{\availablewidth}
	\setlength{\availablewidth}{\linewidth}
	\addtolength{\availablewidth}{-14\tabcolsep} % 7 espacios internos * 2
	\addtolength{\availablewidth}{-0.4pt} % Ancho de líneas verticales

	% =============================================================================
	% LONGTABLE CON CAPTION AL PRINCIPIO
	% =============================================================================
	\begin{longtable}{
		|>{\centering\arraybackslash}m{0.07\availablewidth}			% ESPECIE
		|>{\centering\arraybackslash}m{0.07\availablewidth}           % Nº CASOS
		|>{\centering\arraybackslash}m{0.14\availablewidth}           % ENFERMEDADES
		|>{\raggedright\arraybackslash}m{0.12\availablewidth}          % FÁRMACO
		|>{\raggedright\arraybackslash}m{0.24\availablewidth}          % PRINCIPIO ACTIVO
		|>{\raggedright\arraybackslash}m{0.152\availablewidth}          % DOSIS/VÍA
		|>{\centering\arraybackslash}m{0.06\availablewidth}            % TIEMPO
		|>{\raggedright\arraybackslash}m{0.14\availablewidth}|         % OBSERVACIONES
	}

		% ──────────────── Encabezado y CAPTION COMO TÍTULO (solo en primera página) ────────────────
		\caption{\raggedright Registro de Tratamientos para Enfermedades Bacterianas}
		\label{tab:bacterianas} \\
		\hline
		\rowcolor{blue!10}
		\textbf{ESPECIE} & \textbf{Nº DE CASOS} & \textbf{ENFERMEDADES} & \textbf{FÁRMACO} & 
		\textbf{PRINCIPIO ACTIVO} & \textbf{DOSIS/VÍA DE ADMINISTRACIÓN} & 
		\textbf{DURA\-CIÓN} & \textbf{OBSERVACIONES} \\
		\hline
		\endfirsthead

		% ──────────────── ENCABEZADO PARA PÁGINAS SIGUIENTES ────────────────		
		\multicolumn{8}{l}{\normalsize\textbf{(Continuación) Cuadro \ref{tab:bacterianas}}. Registro de Tratamientos para Enfermedades Virales} \\[6pt]
		\hline
		\rowcolor{blue!10}
		\textbf{ESPECIE} & \textbf{Nº DE CASOS} & \textbf{ENFERMEDADES} & \textbf{FÁRMACO} & 
		\textbf{PRINCIPIO ACTIVO} & \textbf{DOSIS/VÍA DE ADMINISTRACIÓN} & 
		\textbf{DURA\-CIÓN} & \textbf{OBSERVACIONES} \\
		\hline
		\endhead

		% ──────────────── PIE DE PÁGINA (continuación) ────────────────
		\hline
		\multicolumn{8}{r}{\footnotesize\textit{Continúa en la siguiente página \ldots}}
		\endfoot

		% ──────────────── PIE FINAL ────────────────
		\hline
		\multicolumn{8}{l}{\footnotesize\textit{\textbf{Fuente}: Elaboración propia.}}
		\endlastfoot

		% =============================================================================
		% CONTENIDO DE LA TABLA (extraído del PDF '02_bacteriana.pdf')
		% =============================================================================

		% Canino – Traqueo-bronquitis
		\multirow{4}{*}{\textbf{Canino}} & \multirow{4}{*}{15} & \multirow{4}{*}{Traqueo-bronquitis} &
		Complejo yodado & Yoduro de potasio, yodo metálico & 0,3ml/10kg IM & \multirow{4}{*}{3-5 días} & \multirow{4}{*}{Satisfactorio} \\
		\cline{4-6}
		& & & Bromexan & Brohexina & 1-3ml IM & & \\
		\cline{4-6}
		& & & Desalgina & Dipirona & 1ml/10kg IV-IM-SC & & \\
		\cline{4-6}
		& & & Interflox & Erofloxacino & 1ml/10kg IM-SC & & \\
		\hline

		% Canino – Gastroenteritis
		\multirow{4}{*}{\textbf{Canino}} & \multirow{4}{*}{29} & \multirow{4}{*}{Gastroenteritis} &
		Gentamicin & Gentamicina, metilparabeno, propilparabeno & 1ml/20kg IV-IM & \multirow{4}{*}{3-5 días} & \multirow{4}{*}{Satisfactorio} \\
		\cline{4-6}
		& & & AminoLab forte & Aminoacidos, dextrosa, cloruro de potasio, cloruro de sodio, cloruro de calcio, acetato de sodio, vitamina B1 B2 B6 B12 & 1ml/1kg IV & & \\
		\cline{4-6}
		& & & Complejo B & Vitamina B1, Vitamina B2, Vitamina B6, Vitamina B12, Vitamina B15 & 1ml/20kg IM-SC & & \\
		\cline{4-6}
		& & & Hepato-Ject & Acido tioctico, acido orotico, DL-Metionina/N-Acetil-L-Metionina, D-pantenol, Piridoxina HCl, acido folico, cloruro de colina, inotisol, betaina HCl & 1ml/20kg IV-IM & & \\
		\hline

		% Canino – Conjuntivitis
		\textbf{Canino} & 10 & Conjuntivitis & Dexaflox & Colirio, ofloxacina, dexametazona fosfato & 1-2 gotas V.Oftálmica & 3-5 días & Satisfactorio \\
		\hline

		% Canino – Otitis
		\multirow{2}{*}{\textbf{Canino}} & \multirow{2}{*}{7} & \multirow{2}{*}{Otitis} &
		Atriben & Triamcinolona & 1ml/20kg IM-SC & \multirow{2}{*}{3-5 días} & \multirow{2}{*}{Satisfactorio} \\
		\cline{4-6}
		& & & Opter & Enrofloxacina, clotrimazol, betametazona, ivermectina & 6-12 gotas V.Ótica & & \\
		\hline

		% Felino – Traqueo-bronquitis
		\multirow{3}{*}{\textbf{Felino}} & \multirow{3}{*}{3} & \multirow{3}{*}{Traqueo-bronquitis} &
		Complejo yodado & Yoduro de potasio, yodo metálico & 0,3ml/10kg IM & \multirow{3}{*}{3-5 días} & \multirow{3}{*}{Satisfactorio} \\
		\cline{4-6}
		& & & Bromexan & Brohexina & 1-3ml IM & & \\
		\cline{4-6}
		& & & Interflox & Erofloxacino & 1ml/10kg IM-SC & & \\
		\hline

		% Felino – Conjuntivitis
		\textbf{Felino} & 4 & Conjuntivitis & Dexaflox & Colirio, ofloxacina, dexametazona fosfato & 1-2 gotas V.Oftálmica & 3-5 días & Satisfactorio \\
		\hline

		% Felino – Otitis
		\multirow{2}{*}{\textbf{Felino}} & \multirow{2}{*}{1} & \multirow{2}{*}{Otitis} &
		Atriben & Triamcinolona & 1ml/20kg IM-SC & \multirow{2}{*}{3-5 días} & \multirow{2}{*}{Satisfactorio} \\
		\cline{4-6}
		& & & Opter & Enrofloxacina, clotrimazol, betametazona, ivermectina & 6-12 gotas V.Ótica & & \\
		\hline

	\end{longtable}
\end{landscape}