% 05_intoxicaniones.tex
\begin{landscape}

%\section*{REGISTRO DE INTOXICACIONES}
%\vspace{0.3cm}

% =============================================================================
% CONFIGURACIÓN (basada en tu modelo)
% =============================================================================
\setlength{\tabcolsep}{1pt}
\setlength{\extrarowheight}{3pt}
\fontsize{10pt}{10pt}\selectfont

% =============================================================================
	% CÁLCULO DE ANCHOS DE COLUMNAS
	% =============================================================================
	% Total de ancho disponible: \linewidth (en landscape es el ancho horizontal)
	% Restamos el espacio ocupado por líneas verticales y espacios entre columnas
	% 8 columnas = 7 líneas verticales + espacios entre columnas
	%\newlength{\availablewidth}
	\setlength{\availablewidth}{\linewidth}
	\addtolength{\availablewidth}{-14\tabcolsep} % 7 espacios internos * 2
	\addtolength{\availablewidth}{-0.4pt} % Ancho de líneas verticales

% =============================================================================
% LONGTABLE CON CAPTION AL PRINCIPIO
% =============================================================================
\begin{longtable}{
	|>{\centering\arraybackslash}m{0.07\availablewidth}           % ESPECIE
	|>{\centering\arraybackslash}m{0.07\availablewidth}           % Nº CASOS
	|>{\centering\arraybackslash}m{0.14\availablewidth}           % ENFERMEDADES
	|>{\raggedright\arraybackslash}m{0.12\availablewidth}          % FÁRMACO
	|>{\raggedright\arraybackslash}m{0.24\availablewidth}          % PRINCIPIO ACTIVO
	|>{\raggedright\arraybackslash}m{0.152\availablewidth}          % DOSIS/VÍA
	|>{\centering\arraybackslash}m{0.06\availablewidth}            % TIEMPO
	|>{\raggedright\arraybackslash}m{0.14\availablewidth}|         % OBSERVACIONES
}

% ──────────────── CAPTION COMO TÍTULO (solo en primera página) ────────────────
\caption{\raggedright Registro de Tratamientos para Intoxicación Alimenticia}
\label{tab:intoxicaciones} \\
\hline
\rowcolor{blue!10}
\textbf{ESPECIE} & \textbf{Nº DE CASOS} & \textbf{ENFERMEDADES} & \textbf{FÁRMACO} & 
\textbf{PRINCIPIO ACTIVO} & \textbf{DOSIS/VÍA DE ADMINISTRACIÓN} & 
\textbf{DURA\-CIÓN} & \textbf{OBSERVACIONES} \\
\hline
\endfirsthead

% ──────────────── ENCABEZADO PARA PÁGINAS SIGUIENTES ────────────────
\multicolumn{8}{l}{\normalsize\textbf{(Continuación) Cuadro \ref{tab:intoxicaciones}}. Registro de Tratamientos para Intoxicación Alimenticia}\\[6pt]
\hline
\rowcolor{blue!10}
\textbf{ESPECIE} & \textbf{Nº DE CASOS} & \textbf{ENFERMEDADES} & \textbf{FÁRMACO} & 
\textbf{PRINCIPIO ACTIVO} & \textbf{DOSIS/VÍA DE ADMINISTRACIÓN} & 
\textbf{DURA\-CIÓN} & \textbf{OBSERVACIONES} \\
\hline
\endhead

% ──────────────── PIE DE PÁGINA (continuación) ────────────────
\hline
\multicolumn{8}{r}{\footnotesize\textit{Continúa en la siguiente página \ldots}}
\endfoot

% ──────────────── PIE FINAL ────────────────
\hline
\multicolumn{8}{l}{\footnotesize\textit{\textbf{Fuente}: Elaboración propia.}}
\endlastfoot

% =============================================================================
% CONTENIDO DE LA TABLA (extraído del PDF '05_intoxicaniones.pdf')
% =============================================================================

% Canino – Intoxicación alimenticia
\multirow{5}{*}{\textbf{Canino}} & \multirow{5}{*}{13} & \multirow{5}{*}{\parbox{\linewidth}{Intoxicación alimenticia}} &
Suero fisiológico & Cloruro de sodio & Según criterio médico IV & \multirow{5}{*}{3 días} & \multirow{5}{*}{Satisfactorio} \\
\cline{4-6}
& & & Mercepton & Acetil D-L-Metionina, cloruro de colina, Inositol, Vitamina B1 B2 B6 B12, nicotinmida, pantotenato de calcio & 1ml/10kg IV-IM-SC & & \\
\cline{4-6}
& & & Desalgina & Dipirona & 1ml/10kg IV-IM-SC & & \\
\cline{4-6}
& & & Hepato-Ject & Acido tioctico, acido orotico, DL-Metionina/N-Acetil-L-Metionina, D-pantenol, Piridoxina HCl, acido folico, cloruro de colina, inotisol, betaina HCl & 1ml/20kg IV-IM & & \\
\cline{4-6}
& & & Atropina & Sulfato de atropina & 0,5ml/10kg IV-IM-SC & & \\
\hline

\end{longtable}
\end{landscape}