\begin{landscape}

%\subsection*{Cuadro Nº 5: Intoxicaciones}
%\vspace{0.3cm}

% =============================================================================
% CONFIGURACIÓN TIPOGRÁFICA (NO TOCAR – FUNCIONA PERFECTAMENTE)
% =============================================================================
%\fontsize{10pt}{15pt}\selectfont
%\setlength{\extrarowheight}{2pt}
%\setlength{\tabcolsep}{3pt}
\setlength{\tabcolsep}{3pt}
\fontsize{10pt}{10pt}\selectfont
\setlength{\extrarowheight}{2pt}

% =============================================================================
% TABLA
% =============================================================================
\begin{longtable}{
  |>{\centering\arraybackslash}m{0.10\linewidth}
  |>{\centering\arraybackslash}m{0.05\linewidth}
  |>{\centering\arraybackslash}m{0.13\linewidth}
  |>{\raggedright\arraybackslash}m{0.13\linewidth}
  |>{\raggedright\arraybackslash}m{0.18\linewidth}
  |>{\raggedright\arraybackslash}m{0.14\linewidth}
  |>{\centering\arraybackslash}m{0.06\linewidth}
  |>{\raggedright\arraybackslash}m{0.16\linewidth}|
}

% -----------------------------------------------------------------------------
% CAPTION Y LABEL
% -----------------------------------------------------------------------------
\caption[]{Tratamiento de intoxicaciones}
\label{tab:intoxicaciones}\\
\endfirsthead

% -----------------------------------------------------------------------------
% ENCABEZADO (PRIMERA PÁGINA Y SIGUIENTES)
% -----------------------------------------------------------------------------
\hline
\rowcolor{blue!10}
\textbf{ESPECIE} &
\textbf{Nº CASOS} &
\textbf{ENFERMEDAD} &
\textbf{FÁRMACO} &
\textbf{PRINCIPIO ACTIVO} &
\textbf{DOSIS/VÍA de ADMINISTRACIÓN} &
\textbf{TIEMPO} &
\textbf{OBSERVACIONES} \\
\hline
\endhead

% -----------------------------------------------------------------------------
% PIE DE CONTINUACIÓN
% -----------------------------------------------------------------------------
\hline
\multicolumn{8}{r}{\footnotesize\textit{Continúa en la siguiente página}} \\
\endfoot

\endlastfoot

\hline
\rowcolor{blue!10}
\textbf{ESPECIE} &
\textbf{Nº CASOS} &
\textbf{ENFERMEDAD} &
\textbf{FÁRMACO} &
\textbf{PRINCIPIO ACTIVO} &
\textbf{DOSIS/VÍA de ADMINISTRACIÓN} &
\textbf{TIEMPO} &
\textbf{OBSERVACIONES} \\
\hline

% =============================================================================
% CANINO – INTOXICACIÓN ALIMENTICIA
% =============================================================================
\multirow{5}{*}{\textbf{Canino}} &
\multirow{5}{*}{13} &
\multirow{5}{*}{\parbox{\linewidth}{Intoxicación alimenticia}} &
Suero fisiológico & Cloruro de sodio & Según criterio médico IV &
\multirow{5}{*}{3 días} &
\multirow{5}{*}{Satisfactorio} \\
\cline{4-6}
& & & Mercepton & Acetil D-L-Metionina, cloruro de colina, Inositol, Vitamina B1 B2 B6 B12, nicotinamida, pantotenato de calcio & 1\,ml/10\,kg IV-IM-SC & & \\
\cline{4-6}
& & & Desalgina & Dipirona & 1\,ml/10\,kg IV-IM-SC & & \\
\cline{4-6}
& & & Hepato-Ject & Ácido tioctico, ácido orótico, DL-Metionina/N-Acetil-L-Metionina, D-pantenol, Piridoxina HCl, ácido fólico, cloruro de colina, inositol, betaina HCl & 1\,ml/20\,kg IV-IM & & \\
\cline{4-6}
& & & Atropina & Sulfato de atropina & 0,5\,ml/10\,kg IV-IM-SC & & \\
\hline

\end{longtable}
\end{landscape}